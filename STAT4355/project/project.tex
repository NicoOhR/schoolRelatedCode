\documentclass[11pt]{report}

% Basic packages you probably want
\usepackage[margin=1in]{geometry}
\usepackage[T1]{fontenc}
\usepackage[utf8]{inputenc}
\usepackage{lmodern}
\usepackage{float}
\usepackage{amsmath, amssymb}
\usepackage{graphicx}
\usepackage{hyperref}
\usepackage{titlesec}
\renewcommand{\thesection}{\arabic{section}}
\renewcommand{\thesubsection}{\thesection.\arabic{subsection}}
\title{Linear Legands Project Report}
\author{
  N. Ohayon Rozanes \and
  Daisy Rueda  \and
  Minh Huy Nguyen \and
  Illia Myronov
}
\date{\today}

\begin{document}

\maketitle

\tableofcontents
\clearpage

\section{Introduction}
Every college student faces the same challenges: how to divide
limited hours between studying, sleeping, socializing, job searching
and everything in between. Some can manage balanced routines with
ease, while the others struggle to keep up. But how do these everyday
habits shape success and stress?
This project explores that question through a data-driven analysis of
student lifestyle patterns. Using linear modeling, we examine how
variables exercise influence GPA and stress levels.
Ultimately, this study seeks to translate everyday student
experiences into measurable insights that reveal how balance, effort,
and rest define the path to academic success and personal well-being.

\subsection{Background}
This dataset contains 8 columns and 2,000 observations. Every row
is a student with a unique ID, 5 observations about the student’s
habits, the student’s GPA, and the student’s stress level. The row
values are listed below.
\begin{enumerate}
  \item \textbf{ID}: Unique ID assigned to every student. Ranges 1-2000
  \item \textbf{Study Hours Per Day} : Average number of hours a student
    spends studying every day. Ranges 5:10
  \item \textbf{Extracurricular Hours Per Day} : Average number of hours a
    student spends on non:study activities (i.e., clubs) every day. Ranges 0-4
  \item \textbf{Sleep Hours Per Day} : Average number of hours a student
    spends sleeping every day. Ranges 5:10
  \item \textbf{Social Hours Per Day} : Average number of hours a student
    spends socializing every day. Ranges 0:6
  \item \textbf{Physical Activity Hours Per Day} : Average number of hours a
    student spends exercising every day. Ranges 0:13
  \item \textbf{GPA} : Student’s GPA
  \item \textbf{Stress Level} : Reported stress of a student, categorical.
    Obtains values: Low, Moderate, High.
\end{enumerate}

\subsection{Goals}

Firstly, we would like to be able to analyze how a students daily
habits effect their academic performance, as measure by their GPA. We
would also like to analyze the relationship between stress other variables.

\section{Methods}
\subsection{Preprocessing}
Firstly, we drop the ID column, as it does not aid in the analysis.
The data set does not have any missing or censored data so no rows
need to be removed on the basis of missing values. The stress level
variable is categorical, with values low, moderate and high, which we
use ordinal encoding, transforming it into $1,2$ and $3$ respectively.
\subsection{Exploratry Data Analysis}
\begin{center}
  \includegraphics[scale=0.15]{histograms.png}
\end{center}
From the above figure, showing the histograms of the quantitative
variables of the data set, we can see that study hours are fairly
uniform, ranging between $5$ and $10$ hours, extracurricular
involvement ranges from $0$ to $4$ hours without a clear skew in
either direction, sleep hours similarly varies between $5$ and $10$
hours with a similar shape, as does social time, which varied between
$0$ and $6$ hours. Physical activity does exhibit a definite right
bias, and ranges between $0$ to $6$ hours. Notably, extracurricular
activity does have several students far away from the mean, which
could be explained by student athletes, for example. Lastly, GPA
exhibits a bell curve distribution, perhaps individual classes grade
to a curve, or that is simply the natural distribution of the data.

\begin{center}
  \includegraphics[scale=0.4]{stress.png}
\end{center}

Next, looking at the distribution of stress level, we see that by far, the
largest group is the high stress group, with more than one thousand
students reporting high levels of stress, indicating that stress is a
widespread issue in this student population. This prompts further
investigation on the effects of stress and the other variables.

\begin{center}
  \includegraphics[scale=0.2]{stressagainstvariables.png}
\end{center}

Strikingly, every student that studies more than $8$ hours a day
reports high stress level, but is also more positively correlated with
a higher GPA. Additionally, every student that sleeps less than $6$
hours reports high stress, but there is not as clear an indication of
an increase in GPA. Another interesting observation is that the high
stress group has the most varied GPA distribution, but does have the
highest mean compared to the other groups.

\section{Model Selection}

\subsection{Predictor Selection}

An interesting feature of this data set is that the hours columns
always add up to $24$, which causes perfect colinearity with the all
$1$ column added to the data matrix for the intercept. Since the
model would not be interpretable if it was forced to go through the
origin (a student cannot spend time doing nothing), removing the
intercept does not make sense. To avoid colinearity, we have to drop
one of the hours columns. Removing one of the columns changes the
interpretation of the model slightly, and makes it so you have to
interpret the coefficients in reference to the dropped column. Put
concretely, a coefficient of $\beta$ for the study hours predictors,
with the extracurricular hours dropped, should be interpreted as
"spending an hour more on studying than extracurriculars increases
GPA by $\beta$".

We chose to drop the extracurricular columns since, as shown by the
table below, it had the lowest variance out the hours columns, so
students have the most consistent extracurriculars as a baseline.

\begin{table}[H]
  \centering
  \begin{tabular}{l c}
    \hline
    Variable & Variance \\
    \hline
    \textbf{Study Hours Per Day} & 2.02746 \\
    \textbf{Extracurricular Hours Per Day} & 1.33600 \\
    \textbf{Sleep Hours Per Day} & 2.13437 \\
    \textbf{Social Hours Per Day} & 2.85108 \\
    \textbf{Physical Activity Hours Per Day} & 6.32075 \\
    \hline
  \end{tabular}
  \caption{Sample variances of daily hour-allocation variables.}
\end{table}

Then, to select the predictors which were most effective, we used
backwards and forwards stepwise selection, using MSE and explained
variance as the selection criterion. All four stepwise selection
methods, backwards with MSE, backwards with explain variance,
forwards with MSE, and forward with explained variance, concluded
that the model which maximizes the metrics is

$$
\text{GPA} \sim \text{Hours Studied} +  \text{Hours Slept}
$$

The criterion of MSE and explained variance was used because, across
the board, the AIC and BIC were very similar. This model has the
following summary

\begin{table}[H]
  \centering
  \begin{tabular}{l c c c c}
    \hline
    Variable & Estimate & Std. Error & $t$-value & $p$-value \\
    \hline
    Intercept & 1.9999 & 0.0331 & 60.31 & $<0.001$ \\
    Study Hours & 0.1542 & 0.0032 & 48.42 & $<0.001$ \\
    Sleep Hours & $-0.0049$ & 0.0031 & $-1.58$ & 0.115 \\
    \hline
  \end{tabular}
  \caption{OLS regression results for GPA on study and sleep hours ($n=2000$).}
\end{table}

We can see that the $p$ value of the sleep hours predictor is
much higher than the typical $0.05$ significance threshold. We
investigate this further by using a partial F-test to compare the
model using only study hours and the model using study hours and
sleep hours, which yields the following results:

\begin{table}[H]
  \centering
  \begin{tabular}{l c c c c}
    \hline
    Model & Residual df & Residual SS & $F$ & $p$-value \\
    \hline
    GPA $\sim$ Study & 1998 & 82.1277 &  &  \\
    GPA $\sim$ Study + Sleep & 1997 & 82.0257 & 2.4819 & 0.1153 \\
    \hline
  \end{tabular}
  \caption{ANOVA comparison of nested linear models for GPA.}
\end{table}

Clearly, sleep hours do not provide any statistically significant
improvement in explanatory power beyond study hours alone. This is
further confirmed by the fact that the adjusted $R^2$ is the same for
both models, at $\approx 0.54$. Thus, we come to a final model:

$$
\text{GPA} = 1.964228 + 0.154061\cdot\text{Hours Studied Per Day}
$$

\subsection{Model Summary}

For the final model above, we have the following model fit statistics:

\begin{table}[H]
  \centering
  \begin{tabular}{l c}
    \hline
    Statistic & Value \\
    \hline
    Residual Standard Error & 0.2027 \\
    $R^2$ & 0.5394 \\
    Adjusted $R^2$ & 0.5392 \\
    F-statistic & 2340 \\
    $p$-value & $< 0.001$ \\
    \hline
  \end{tabular}
  \caption{Model fit statistics for the GPA regression model.}
\end{table}

The incredibly small $p$ value indicates that the model is
statistically significant. The coefficient of determination indicates
that approximately $54\%$ of the variability in GPA is accounted for
by the hours studied. Further, the very similar adjusted $R^2$
indicates that the explanatory power of the model is not a result of
overfitting. Further the residual standard error of $0.2027$ suggests
that the on average, the model deviates from the actual values by
about $0.2$ GPA points, which, as a percentage of the overall scale,
is only $0.05\%$.

\subsection{Residual Analysis}

Using the model found above, we plot the histogram and QQ plot of the
residuals to verify that the errors of the data is roughly normally distributed.

\begin{center}
  \includegraphics[scale=0.25]{residualHistogram.png}
  \includegraphics[scale=0.25]{residualQQ.png}
\end{center}

There is a slight left skew to the histogram, indicating that the
model is more likely to underestimate a student's GPA rather than
overestimate it. However, the residuals are centered at around zero
and are in the range of $-1$ to $1$, verifying that the normality
assumptions are well satisfied. The QQ plot further reinforces this,
and shows that the residuals lie very close to the normal reference
line, with only minor deviations towards the extremes. Together,
these plots show that the model errors are approximately normally
distributed with constant variance and no systematic structure which
is unexplained.

\subsection{Prediction and Confidence Intervals}

text

\subsection{Influential Data Points}

We now inspect our data for possibly influential data points. To
start with, we graph the residuals of each data point using the selected model

\begin{center}
  \includegraphics[scale=0.5]{residual_bars.png}
\end{center}

The high number of data points above the cutoffs for the studentized
residuals, with very few standardized or R-student exceedances
indicate mild, leverage-adjusted deviations, and do not indicate, on
their own at least, outliers or model failure. A point of interest is
indexed 702, which has a residual higher than standardized residual
and studentized residual. This point comes up again when we plot the
various influential matrices.

\begin{center}
  \includegraphics[scale=0.5]{influence.png}
\end{center}

Pulling out this specific student, we see that the student's GPA is
much higher than what the model would predict. The point has ordinary
leverage, as indicated by the hat value, but a large residual, which
produces the high Bonferroni-significant R-student residual and a
moderate Cook's distance. This suggests the presence of unmodeled
factors which effect GPA for this individual specifically, than a
structural problem with the model. Additionally, due to the very high
$n$, the rule of thumb bands used for the points that need to be
inspected is very small. For example, the cutoff for Cook's distance
is $4/n = 0.002$, and the maximum value observed for Cook's distance
is $0.0096$, corresponding to observation $702$, which exceeds the
screening line, but is still tiny in absolute terms. Similarly, the
COVRATIO has the band $1 \pm 3p/n = 1 \pm 0.0075$, with the minimum
and maximum points $0.96807$ and $1.00860$, which only very narrowly
exceeds the cutoff. The number of points which are considered extreme
by the diagnostic measure is presented by the table below:

\begin{table}[h]
  \centering
  \begin{tabular}{l c}
    \hline
    \textbf{Diagnostic} & \textbf{Count (Percent)} \\
    \hline
    High leverage $(>2p/n)$ & 24 (1.20\%) \\
    High leverage $(>3p/n)$ & 1 (0.05\%) \\
    High Cook's distance $(>4/n)$ & 92 (4.60\%) \\
    High leverage and high Cook's D & 3 (0.15\%) \\
    Standardized residuals $(|r_i| > 3)$ & 3 (0.15\%) \\
    \hline
  \end{tabular}
  \caption{Leverage, influence, and residual diagnostics for the GPA
  regression model.}
\end{table}

Largely, no values need to be removed since using leverage, Cook's
distance, DFBETAS, and COVRATIO, we found no evidence of undue
influence, as all values are relatively small in absolute terms.

\section{Analysing Stress Level}

\section{Results}
\subsection{Main result}
Some text.

\subsection{Conclusion}

Some text.

\subsection{Future Plans}

\end{document}
