\documentclass[11pt]{report}

% Basic packages you probably want
\usepackage[margin=1in]{geometry}
\usepackage[T1]{fontenc}
\usepackage[utf8]{inputenc}
\usepackage{lmodern}
\usepackage{amsmath, amssymb}
\usepackage{graphicx}
\usepackage{hyperref}
\usepackage{titlesec}
\renewcommand{\thesection}{\arabic{section}}
\renewcommand{\thesubsection}{\thesection.\arabic{subsection}}
\title{Linear Legands Project Report}
\author{
  N. Ohayon Rozanes \and
  Daisy Rueda  \and
  Minh Huy Nguyen \and
  Illia Myronov
}
\date{\today}

\begin{document}

\maketitle

\tableofcontents
\clearpage

\section{Introduction}
Every college student faces the same challenges: how to divide
limited hours between studying, sleeping, socializing, job searching
and everything in between. Some can manage balanced routines with
ease, while the others struggle to keep up. But how do these everyday
habits shape success and stress?
This project explores that question through a data-driven analysis of
student lifestyle patterns. Using linear modeling, we examine how
variables exercise influence GPA and stress levels.
Ultimately, this study seeks to translate everyday student
experiences into measurable insights that reveal how balance, effort,
and rest define the path to academic success and personal well-being.

\subsection{Background}
This dataset contains 8 columns and 2,000 observations. Every row
is a student with a unique ID, 5 observations about the student’s
habits, the student’s GPA, and the student’s stress level. The row
values are listed below.
\begin{enumerate}
  \item \textbf{ID}: Unique ID assigned to every student. Ranges 1-2000
  \item \textbf{Study Hours Per Day} : Average number of hours a student
    spends studying every day. Ranges 5:10
  \item \textbf{Extracurricular Hours Per Day} : Average number of hours a
    student spends on non:study activities (i.e., clubs) every day. Ranges 0-4
  \item \textbf{Sleep Hours Per Day} : Average number of hours a student
    spends sleeping every day. Ranges 5:10
  \item \textbf{Social Hours Per Day} : Average number of hours a student
    spends socializing every day. Ranges 0:6
  \item \textbf{Physical Activity Hours Per Day} : Average number of hours a
    student spends exercising every day. Ranges 0:13
  \item \textbf{GPA} : Student’s GPA
  \item \textbf{Stress Level} : Reported stress of a student, categorical.
    Obtains values: Low, Moderate, High.
\end{enumerate}

\subsection{Goals}

Firstly, we would like to be able to analyze how a students daily
habits effect their academic performance, as measure by their GPA. We
would also like to analyze the relationship between stress other variables.

\section{Methods}
\subsection{Preprocessing}
Firstly, we drop the ID column, as it does not aid in the analysis.
The data set does not have any missing or censored data so no rows
need to be removed on the basis of missing values. The stress level
variable is categorical, with values low, moderate and high, which we
use ordinal encoding, transforming it into $1,2$ and $3$ respectively.
\subsection{Exploratry Data Analysis}
\begin{center}
  \includegraphics[scale=0.15]{histograms.png}
\end{center}
From the above figure, showing the histograms of the quantitative
variables of the data set, we can see that study hours are fairly
uniform, ranging between $5$ and $10$ hours, extracurricular
involvement ranges from $0$ to $4$ hours without a clear skew in
either direction, sleep hours similarily varies between $5$ and $10$
hours with a similar shape, as does social time, which varied between
$0$ and $6$ hours. Physical activity does exhibit a definite right
bias, and ranges between $0$ to $6$ hours. Notably, extracuricular
activity does have several students far away from the mean, which
could be explained by student atheletes, for example. Lastly, GPA
exhibits a bell curve distribution, perhaps individual classes grade
to a curve, or that is simply the natural distribution of the data.

\begin{center}
  \includegraphics[scale=0.4]{stress.png}
\end{center}

Next, looking at the distribution of stress level, we see that by far, the
largest group is the high stress group, with more than one thousand
students reporting high levels of stress, indicating that stress is a
widespread issue in this student population. This prompts further
investigation on the effects of stress and the other variables.

\begin{center}
  \includegraphics[scale=0.2]{stressagainstvariables.png}
\end{center}

Strikingly, every student that studies more than $8$ hours a day
reports high stress level, but is also more positevly correlated with
a higher GPA. Additionaly, every student that sleeps less than $6$
hours reports high stress, but there is not as clear an indication of
an increase in GPA. Another interesting observation is that the high
stress group has the most varied GPA distribution, but does have the
highest mean compared to the other groups.
\section{Results}
\subsection{Main result}
Some text.

\subsection{Conclusion}
Some text.

\end{document}
