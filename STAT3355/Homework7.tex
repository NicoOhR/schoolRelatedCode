% Options for packages loaded elsewhere
\PassOptionsToPackage{unicode}{hyperref}
\PassOptionsToPackage{hyphens}{url}
%
\documentclass[
]{article}
\usepackage{amsmath,amssymb}
\usepackage{iftex}
\ifPDFTeX
  \usepackage[T1]{fontenc}
  \usepackage[utf8]{inputenc}
  \usepackage{textcomp} % provide euro and other symbols
\else % if luatex or xetex
  \usepackage{unicode-math} % this also loads fontspec
  \defaultfontfeatures{Scale=MatchLowercase}
  \defaultfontfeatures[\rmfamily]{Ligatures=TeX,Scale=1}
\fi
\usepackage{lmodern}
\ifPDFTeX\else
  % xetex/luatex font selection
\fi
% Use upquote if available, for straight quotes in verbatim environments
\IfFileExists{upquote.sty}{\usepackage{upquote}}{}
\IfFileExists{microtype.sty}{% use microtype if available
  \usepackage[]{microtype}
  \UseMicrotypeSet[protrusion]{basicmath} % disable protrusion for tt fonts
}{}
\makeatletter
\@ifundefined{KOMAClassName}{% if non-KOMA class
  \IfFileExists{parskip.sty}{%
    \usepackage{parskip}
  }{% else
    \setlength{\parindent}{0pt}
    \setlength{\parskip}{6pt plus 2pt minus 1pt}}
}{% if KOMA class
  \KOMAoptions{parskip=half}}
\makeatother
\usepackage{xcolor}
\usepackage[margin=1in]{geometry}
\usepackage{color}
\usepackage{fancyvrb}
\newcommand{\VerbBar}{|}
\newcommand{\VERB}{\Verb[commandchars=\\\{\}]}
\DefineVerbatimEnvironment{Highlighting}{Verbatim}{commandchars=\\\{\}}
% Add ',fontsize=\small' for more characters per line
\usepackage{framed}
\definecolor{shadecolor}{RGB}{248,248,248}
\newenvironment{Shaded}{\begin{snugshade}}{\end{snugshade}}
\newcommand{\AlertTok}[1]{\textcolor[rgb]{0.94,0.16,0.16}{#1}}
\newcommand{\AnnotationTok}[1]{\textcolor[rgb]{0.56,0.35,0.01}{\textbf{\textit{#1}}}}
\newcommand{\AttributeTok}[1]{\textcolor[rgb]{0.13,0.29,0.53}{#1}}
\newcommand{\BaseNTok}[1]{\textcolor[rgb]{0.00,0.00,0.81}{#1}}
\newcommand{\BuiltInTok}[1]{#1}
\newcommand{\CharTok}[1]{\textcolor[rgb]{0.31,0.60,0.02}{#1}}
\newcommand{\CommentTok}[1]{\textcolor[rgb]{0.56,0.35,0.01}{\textit{#1}}}
\newcommand{\CommentVarTok}[1]{\textcolor[rgb]{0.56,0.35,0.01}{\textbf{\textit{#1}}}}
\newcommand{\ConstantTok}[1]{\textcolor[rgb]{0.56,0.35,0.01}{#1}}
\newcommand{\ControlFlowTok}[1]{\textcolor[rgb]{0.13,0.29,0.53}{\textbf{#1}}}
\newcommand{\DataTypeTok}[1]{\textcolor[rgb]{0.13,0.29,0.53}{#1}}
\newcommand{\DecValTok}[1]{\textcolor[rgb]{0.00,0.00,0.81}{#1}}
\newcommand{\DocumentationTok}[1]{\textcolor[rgb]{0.56,0.35,0.01}{\textbf{\textit{#1}}}}
\newcommand{\ErrorTok}[1]{\textcolor[rgb]{0.64,0.00,0.00}{\textbf{#1}}}
\newcommand{\ExtensionTok}[1]{#1}
\newcommand{\FloatTok}[1]{\textcolor[rgb]{0.00,0.00,0.81}{#1}}
\newcommand{\FunctionTok}[1]{\textcolor[rgb]{0.13,0.29,0.53}{\textbf{#1}}}
\newcommand{\ImportTok}[1]{#1}
\newcommand{\InformationTok}[1]{\textcolor[rgb]{0.56,0.35,0.01}{\textbf{\textit{#1}}}}
\newcommand{\KeywordTok}[1]{\textcolor[rgb]{0.13,0.29,0.53}{\textbf{#1}}}
\newcommand{\NormalTok}[1]{#1}
\newcommand{\OperatorTok}[1]{\textcolor[rgb]{0.81,0.36,0.00}{\textbf{#1}}}
\newcommand{\OtherTok}[1]{\textcolor[rgb]{0.56,0.35,0.01}{#1}}
\newcommand{\PreprocessorTok}[1]{\textcolor[rgb]{0.56,0.35,0.01}{\textit{#1}}}
\newcommand{\RegionMarkerTok}[1]{#1}
\newcommand{\SpecialCharTok}[1]{\textcolor[rgb]{0.81,0.36,0.00}{\textbf{#1}}}
\newcommand{\SpecialStringTok}[1]{\textcolor[rgb]{0.31,0.60,0.02}{#1}}
\newcommand{\StringTok}[1]{\textcolor[rgb]{0.31,0.60,0.02}{#1}}
\newcommand{\VariableTok}[1]{\textcolor[rgb]{0.00,0.00,0.00}{#1}}
\newcommand{\VerbatimStringTok}[1]{\textcolor[rgb]{0.31,0.60,0.02}{#1}}
\newcommand{\WarningTok}[1]{\textcolor[rgb]{0.56,0.35,0.01}{\textbf{\textit{#1}}}}
\usepackage{graphicx}
\makeatletter
\def\maxwidth{\ifdim\Gin@nat@width>\linewidth\linewidth\else\Gin@nat@width\fi}
\def\maxheight{\ifdim\Gin@nat@height>\textheight\textheight\else\Gin@nat@height\fi}
\makeatother
% Scale images if necessary, so that they will not overflow the page
% margins by default, and it is still possible to overwrite the defaults
% using explicit options in \includegraphics[width, height, ...]{}
\setkeys{Gin}{width=\maxwidth,height=\maxheight,keepaspectratio}
% Set default figure placement to htbp
\makeatletter
\def\fps@figure{htbp}
\makeatother
\setlength{\emergencystretch}{3em} % prevent overfull lines
\providecommand{\tightlist}{%
  \setlength{\itemsep}{0pt}\setlength{\parskip}{0pt}}
\setcounter{secnumdepth}{-\maxdimen} % remove section numbering
\ifLuaTeX
  \usepackage{selnolig}  % disable illegal ligatures
\fi
\IfFileExists{bookmark.sty}{\usepackage{bookmark}}{\usepackage{hyperref}}
\IfFileExists{xurl.sty}{\usepackage{xurl}}{} % add URL line breaks if available
\urlstyle{same}
\hypersetup{
  pdftitle={Homework 7},
  pdfauthor={Nimrod Ohayon Rozanes},
  hidelinks,
  pdfcreator={LaTeX via pandoc}}

\title{Homework 7}
\author{Nimrod Ohayon Rozanes}
\date{2024-04-15}

\begin{document}
\maketitle

\section{Question 1}\label{question-1}

\begin{Shaded}
\begin{Highlighting}[]
\NormalTok{cola }\OtherTok{\textless{}{-}} \FunctionTok{c}\NormalTok{(}\FloatTok{15.997}\NormalTok{, }\FloatTok{16.005}\NormalTok{, }\FloatTok{15.981}\NormalTok{, }\FloatTok{15.954}\NormalTok{, }\FloatTok{15.986}\NormalTok{, }\FloatTok{16.021}\NormalTok{, }\FloatTok{15.985}\NormalTok{, }\FloatTok{16.001}\NormalTok{, }\FloatTok{16.018}\NormalTok{, }\FloatTok{16.056}\NormalTok{)}

\CommentTok{\#step 1 specify distro }
\CommentTok{\# Cola \textasciitilde{} Norm}

\CommentTok{\#step 2 null and alt}
\CommentTok{\#H\_0 = fill == 16}
\CommentTok{\#H\_a = fill != 16}

\CommentTok{\#step 3 t stat}
\NormalTok{X\_bar }\OtherTok{\textless{}{-}}\NormalTok{ (}\FunctionTok{mean}\NormalTok{(cola) }\SpecialCharTok{{-}} \DecValTok{16}\NormalTok{)}\SpecialCharTok{/}\NormalTok{(}\FunctionTok{sd}\NormalTok{(cola)}\SpecialCharTok{/}\FunctionTok{sqrt}\NormalTok{(}\FunctionTok{length}\NormalTok{(cola)))}

\NormalTok{p\_value }\OtherTok{\textless{}{-}} \FunctionTok{pnorm}\NormalTok{(X\_bar)}

\NormalTok{p\_value}
\end{Highlighting}
\end{Shaded}

\begin{verbatim}
## [1] 0.5183085
\end{verbatim}

\begin{Shaded}
\begin{Highlighting}[]
\FunctionTok{t.test}\NormalTok{(cola)}
\end{Highlighting}
\end{Shaded}

\begin{verbatim}
## 
##  One Sample t-test
## 
## data:  cola
## t = 1836.4, df = 9, p-value < 2.2e-16
## alternative hypothesis: true mean is not equal to 0
## 95 percent confidence interval:
##  15.98069 16.02011
## sample estimates:
## mean of x 
##   16.0004
\end{verbatim}

in either approach, the p value is higher than the signficance level we
set, and therefore we fail to reject the null hypothesis, and conclude
that the cola bottles are filled with 16oz.

\section{Question 2}\label{question-2}

\begin{Shaded}
\begin{Highlighting}[]
\NormalTok{stressed }\OtherTok{\textless{}{-}} \DecValTok{130}

\NormalTok{tot }\OtherTok{\textless{}{-}} \DecValTok{200} 

\NormalTok{prop }\OtherTok{\textless{}{-}}\NormalTok{ stressed}\SpecialCharTok{/}\NormalTok{tot}

\NormalTok{P\_0 }\OtherTok{\textless{}{-}} \FloatTok{0.7}

\NormalTok{std\_e }\OtherTok{\textless{}{-}} \FunctionTok{sqrt}\NormalTok{(P\_0 }\SpecialCharTok{*}\NormalTok{ (}\DecValTok{1} \SpecialCharTok{{-}}\NormalTok{ P\_0) }\SpecialCharTok{/}\NormalTok{ tot)}

\NormalTok{Z }\OtherTok{\textless{}{-}}\NormalTok{ (prop }\SpecialCharTok{{-}}\NormalTok{ P\_0)}\SpecialCharTok{/}\NormalTok{std\_e}

\NormalTok{p\_value }\OtherTok{\textless{}{-}} \DecValTok{2} \SpecialCharTok{*}\NormalTok{ (}\DecValTok{1} \SpecialCharTok{{-}} \FunctionTok{pnorm}\NormalTok{(}\FunctionTok{abs}\NormalTok{(Z)))}

\NormalTok{p\_value}
\end{Highlighting}
\end{Shaded}

\begin{verbatim}
## [1] 0.1228226
\end{verbatim}

\begin{Shaded}
\begin{Highlighting}[]
\FunctionTok{prop.test}\NormalTok{(}\AttributeTok{x =} \DecValTok{130}\NormalTok{, }\AttributeTok{n =} \DecValTok{200}\NormalTok{, }\AttributeTok{p =} \FloatTok{0.7}\NormalTok{, }\AttributeTok{alternative =} \StringTok{"two.sided"}\NormalTok{, }\AttributeTok{conf.level =} \FloatTok{0.95}\NormalTok{)}
\end{Highlighting}
\end{Shaded}

\begin{verbatim}
## 
##  1-sample proportions test with continuity correction
## 
## data:  130 out of 200, null probability 0.7
## X-squared = 2.1488, df = 1, p-value = 0.1427
## alternative hypothesis: true p is not equal to 0.7
## 95 percent confidence interval:
##  0.5790769 0.7150579
## sample estimates:
##    p 
## 0.65
\end{verbatim}

We fail to reject the null hypothesis, and conclude that the proportion
of students on this campus who are stressed is not entirely different
from 70\%

\section{Question 3}\label{question-3}

\begin{Shaded}
\begin{Highlighting}[]
\NormalTok{mean\_under\_50 }\OtherTok{\textless{}{-}} \DecValTok{2}
\NormalTok{mean\_over }\OtherTok{\textless{}{-}} \FloatTok{1.85}
\NormalTok{std\_under\_50 }\OtherTok{\textless{}{-}} \FloatTok{0.812}
\NormalTok{std\_over }\OtherTok{\textless{}{-}} \FloatTok{0.837}
\NormalTok{n\_under\_50 }\OtherTok{\textless{}{-}} \DecValTok{350}
\NormalTok{n\_over }\OtherTok{\textless{}{-}} \DecValTok{150}

\NormalTok{se\_18\_50 }\OtherTok{\textless{}{-}}\NormalTok{ std\_under\_50 }\SpecialCharTok{/} \FunctionTok{sqrt}\NormalTok{(n\_under\_50)}
\NormalTok{se\_over\_50 }\OtherTok{\textless{}{-}}\NormalTok{ std\_over }\SpecialCharTok{/} \FunctionTok{sqrt}\NormalTok{(n\_over)}
\NormalTok{combined\_se }\OtherTok{\textless{}{-}} \FunctionTok{sqrt}\NormalTok{(se\_18\_50}\SpecialCharTok{\^{}}\DecValTok{2} \SpecialCharTok{+}\NormalTok{ se\_over\_50}\SpecialCharTok{\^{}}\DecValTok{2}\NormalTok{)}

\NormalTok{t }\OtherTok{\textless{}{-}}\NormalTok{ (mean\_over }\SpecialCharTok{{-}}\NormalTok{ mean\_under\_50)}\SpecialCharTok{/}\NormalTok{combined\_se}

\NormalTok{p\_value }\OtherTok{\textless{}{-}} \FunctionTok{pt}\NormalTok{(t, }\AttributeTok{df =}\NormalTok{ n\_under\_50 }\SpecialCharTok{+}\NormalTok{ n\_over }\SpecialCharTok{{-}} \DecValTok{2}\NormalTok{, }\AttributeTok{lower.tail =} \ConstantTok{TRUE}\NormalTok{)}

\NormalTok{p\_value}
\end{Highlighting}
\end{Shaded}

\begin{verbatim}
## [1] 0.03225135
\end{verbatim}

\begin{Shaded}
\begin{Highlighting}[]
\CommentTok{\#generating samples for t.test to use}
\FunctionTok{set.seed}\NormalTok{(}\DecValTok{123}\NormalTok{)}
\NormalTok{data\_under\_50 }\OtherTok{\textless{}{-}} \FunctionTok{rnorm}\NormalTok{(}\AttributeTok{n =} \DecValTok{350}\NormalTok{, }\AttributeTok{mean =} \DecValTok{2}\NormalTok{, }\AttributeTok{sd =} \FloatTok{0.812}\NormalTok{)}
\NormalTok{data\_over }\OtherTok{\textless{}{-}} \FunctionTok{rnorm}\NormalTok{(}\AttributeTok{n =} \DecValTok{150}\NormalTok{, }\AttributeTok{mean =} \FloatTok{1.85}\NormalTok{, }\AttributeTok{sd =} \FloatTok{0.837}\NormalTok{)}


\FunctionTok{t.test}\NormalTok{(}\AttributeTok{x =}\NormalTok{ data\_under\_50, }\AttributeTok{y =}\NormalTok{ data\_over,}
                      \AttributeTok{alternative =} \StringTok{"greater"}\NormalTok{, }
                      \AttributeTok{var.equal =} \ConstantTok{TRUE}\NormalTok{)}
\end{Highlighting}
\end{Shaded}

\begin{verbatim}
## 
##  Two Sample t-test
## 
## data:  data_under_50 and data_over
## t = 1.7441, df = 498, p-value = 0.04088
## alternative hypothesis: true difference in means is greater than 0
## 95 percent confidence interval:
##  0.00749825        Inf
## sample estimates:
## mean of x mean of y 
##  2.024209  1.888279
\end{verbatim}

Despite the limitation of not having access to the original vectors,
both p values are under the 5\%, leading us to accept the alternative
hypothesis and conclude that there is a signficant different between the
usage of people over 50 and those under.

\section{Question 4}\label{question-4}

\begin{Shaded}
\begin{Highlighting}[]
\NormalTok{apple\_returns }\OtherTok{\textless{}{-}} \DecValTok{14}
\NormalTok{apple\_sales }\OtherTok{\textless{}{-}} \DecValTok{150}
\NormalTok{samsung\_returns }\OtherTok{\textless{}{-}} \DecValTok{15}
\NormalTok{samsung\_sales }\OtherTok{\textless{}{-}} \DecValTok{125}

\NormalTok{p\_apple }\OtherTok{\textless{}{-}}\NormalTok{ apple\_returns }\SpecialCharTok{/}\NormalTok{ apple\_sales}
\NormalTok{p\_samsung }\OtherTok{\textless{}{-}}\NormalTok{ samsung\_returns }\SpecialCharTok{/}\NormalTok{ samsung\_sales}
\NormalTok{p\_pooled }\OtherTok{\textless{}{-}}\NormalTok{ (apple\_returns }\SpecialCharTok{+}\NormalTok{ samsung\_returns) }\SpecialCharTok{/}\NormalTok{ (apple\_sales }\SpecialCharTok{+}\NormalTok{ samsung\_sales)}

\NormalTok{se }\OtherTok{\textless{}{-}} \FunctionTok{sqrt}\NormalTok{(p\_pooled }\SpecialCharTok{*}\NormalTok{ (}\DecValTok{1} \SpecialCharTok{{-}}\NormalTok{ p\_pooled) }\SpecialCharTok{*}\NormalTok{ (}\DecValTok{1}\SpecialCharTok{/}\NormalTok{apple\_sales }\SpecialCharTok{+} \DecValTok{1}\SpecialCharTok{/}\NormalTok{samsung\_sales))}

\NormalTok{Z }\OtherTok{\textless{}{-}}\NormalTok{ (p\_apple }\SpecialCharTok{{-}}\NormalTok{ p\_samsung) }\SpecialCharTok{/}\NormalTok{ se}

\NormalTok{p\_value }\OtherTok{\textless{}{-}} \FunctionTok{pnorm}\NormalTok{(Z) }

\NormalTok{p\_value}
\end{Highlighting}
\end{Shaded}

\begin{verbatim}
## [1] 0.2367125
\end{verbatim}

\begin{Shaded}
\begin{Highlighting}[]
\FunctionTok{prop.test}\NormalTok{(}\AttributeTok{x =} \FunctionTok{c}\NormalTok{(apple\_returns, samsung\_returns),}
                         \AttributeTok{n =} \FunctionTok{c}\NormalTok{(apple\_sales, samsung\_sales),}
                         \AttributeTok{alternative =} \StringTok{"less"}\NormalTok{,}
                         \AttributeTok{correct =} \ConstantTok{FALSE}\NormalTok{)}
\end{Highlighting}
\end{Shaded}

\begin{verbatim}
## 
##  2-sample test for equality of proportions without continuity correction
## 
## data:  c(apple_returns, samsung_returns) out of c(apple_sales, samsung_sales)
## X-squared = 0.51397, df = 1, p-value = 0.2367
## alternative hypothesis: less
## 95 percent confidence interval:
##  -1.00000000  0.03507449
## sample estimates:
##     prop 1     prop 2 
## 0.09333333 0.12000000
\end{verbatim}

The test returns a p value under 5\%, leading us to conclude that
samsung devices are returned at a higher rate than apple products.

\section{Question 5}\label{question-5}

\begin{Shaded}
\begin{Highlighting}[]
\FunctionTok{library}\NormalTok{(UsingR)}
\end{Highlighting}
\end{Shaded}

\begin{verbatim}
## Loading required package: MASS
\end{verbatim}

\begin{verbatim}
## Loading required package: HistData
\end{verbatim}

\begin{verbatim}
## Loading required package: Hmisc
\end{verbatim}

\begin{verbatim}
## 
## Attaching package: 'Hmisc'
\end{verbatim}

\begin{verbatim}
## The following objects are masked from 'package:base':
## 
##     format.pval, units
\end{verbatim}

\begin{Shaded}
\begin{Highlighting}[]
\NormalTok{age }\OtherTok{\textless{}{-}}\NormalTok{ babies}\SpecialCharTok{$}\NormalTok{age[babies}\SpecialCharTok{$}\NormalTok{age }\SpecialCharTok{!=} \DecValTok{99}\NormalTok{]}
\NormalTok{dage }\OtherTok{\textless{}{-}}\NormalTok{ babies}\SpecialCharTok{$}\NormalTok{dage[babies}\SpecialCharTok{$}\NormalTok{dage }\SpecialCharTok{!=} \DecValTok{99}\NormalTok{]}

\NormalTok{n\_mom }\OtherTok{\textless{}{-}} \FunctionTok{length}\NormalTok{(age)}
\NormalTok{n\_dad }\OtherTok{\textless{}{-}} \FunctionTok{length}\NormalTok{(dage)}

\NormalTok{mean\_mom }\OtherTok{\textless{}{-}} \FunctionTok{mean}\NormalTok{(age)}
\NormalTok{mean\_dad }\OtherTok{\textless{}{-}} \FunctionTok{mean}\NormalTok{(dage)}

\NormalTok{var\_mom }\OtherTok{\textless{}{-}} \FunctionTok{var}\NormalTok{(age)}
\NormalTok{var\_dad }\OtherTok{\textless{}{-}} \FunctionTok{var}\NormalTok{(dage)}

\NormalTok{se }\OtherTok{\textless{}{-}} \FunctionTok{sqrt}\NormalTok{(mean\_mom}\SpecialCharTok{/}\NormalTok{n\_mom }\SpecialCharTok{+}\NormalTok{ mean\_dad}\SpecialCharTok{/}\NormalTok{n\_dad)}

\NormalTok{t }\OtherTok{\textless{}{-}}\NormalTok{ (mean\_mom }\SpecialCharTok{{-}}\NormalTok{ mean\_dad) }\SpecialCharTok{/}\NormalTok{ se}

\NormalTok{p\_value }\OtherTok{=} \DecValTok{2} \SpecialCharTok{*} \FunctionTok{pt}\NormalTok{(}\SpecialCharTok{{-}}\FunctionTok{abs}\NormalTok{(t), }\AttributeTok{df =} \FunctionTok{min}\NormalTok{(n\_mom, n\_dad) }\SpecialCharTok{{-}} \DecValTok{1}\NormalTok{)}

\FunctionTok{t.test}\NormalTok{(age, dage, }\AttributeTok{alternative =} \StringTok{"two.sided"}\NormalTok{, }\AttributeTok{var.equal =} \ConstantTok{TRUE}\NormalTok{)}
\end{Highlighting}
\end{Shaded}

\begin{verbatim}
## 
##  Two Sample t-test
## 
## data:  age and dage
## t = -12.157, df = 2461, p-value < 2.2e-16
## alternative hypothesis: true difference in means is not equal to 0
## 95 percent confidence interval:
##  -3.591901 -2.594065
## sample estimates:
## mean of x mean of y 
##  27.25527  30.34825
\end{verbatim}

There is a very low p-value, indicating a very strong difference between
the ages of the mom and the dad.

\end{document}
