\documentclass{article} % This command is used to set the type of
% document you are working on such as an article, book, or presenation
\usepackage{
  amsmath,      % Math Environments
  amssymb,      % Extended Symbols
  enumerate,        % Enumerate Environments
  graphicx,      % Include Images
  lastpage,      % Reference Lastpage
  multicol,      % Use Multi-columns
  multirow,      % Use Multi-rows
  cases,
}

\usepackage{geometry} % This package allows the editing of the page layout
\usepackage{amsmath}  % This package allows the use of a large range
% of mathematical formula, commands, and symbols
\usepackage{graphicx}  % This package allows the importing of images

\newcommand{\question}[2][]{
  \begin{flushleft}
    \textbf{Question #1}: \textit{#2}

\end{flushleft}}
\newcommand{\sol}{\textbf{Solution}:} %Use if you want a boldface solution line
\newcommand{\maketitletwo}[2][]{
  \begin{center}
    \Large{\textbf{Assignment #1}

    Mathematical Statistics} % Name of course here
    \vspace{5pt}

    \normalsize{N. Ohayon Rozanes  % Your name here

    \today}        % Change to due date if preferred
    \vspace{15pt}

\end{center}}
\begin{document}
\maketitletwo[1]  % Optional argument is assignment number
%Keep a blank space between maketitletwo and \question[1]

\question[1]{}
\begin{enumerate}[(a)]
  \item
    \[
      f_X(x) = \frac12 e^{-|x|} \quad -\infty < x < \infty \quad Y = |X|^3
    \]
    On the interval $A_1 := (-\infty, 0]$ the transformation $Y =
    |X|^3$ is monotonically decreasing, similarily on the interval $
    A_2 := [0, \infty)$ the transformation is monotonically increasing.
    Once $\mathcal{X}$ is partitioned to $A_1, A_2$. Further define
    \[
      g_1(x) = -x^3 \quad g_2(x) = x^3
    \]
    And their inverses:
    \[
      g_1^{-1}(y) = -(y)^\frac13 \quad g_2^{-1}(y) = y^\frac13
    \]
    Which satisfy theorem 2.1.8, thus:
    \begin{align*}
      f_Y(y) &= \sum_{i=1}^k f_X(g^{-1}_i(y))
      \left|\frac{d}{dy}g_i^{-1}(y)\right|\\ &=
      \frac12e^{-|(y)^\frac13|}\left|\frac13y^{-\frac23} \right| +
      \frac12e^{-|y^\frac13|}\left|\frac13y^{-\frac23}\right| \\
      &= \frac13y^{-\frac23}e^{-|y^\frac13|}
    \end{align*}
    Also, set $u := y^\frac{1}3$
    \begin{align*}
      \int_0^\infty f_Y(y) dy &= \int_0^\infty
      \frac13y^{-\frac23}e^{-|y^\frac13|} dy \\
      &= \int_0^\infty e^{-|u|} du \\
      &= \int_0^\infty e^{-u} du \\
      &= -e^{-u}\Big|_0^\infty \\
      &= 0 + 1 \\
      &= 1
    \end{align*}
  \item
    \[
      f_X(x) = \frac38(x+1)^2 \quad -1 < x < 1 \quad Y = 1 - X^2
    \]
    The transformation $g(x) = 1 - x^2$ is monotonically increasing on
    $A_1 := [-1, 0]$, and monotonically decreasing on $A_2 := [0,
    1]$, thus to find $f_Y(y)$, define:
    \[
      g_1(x) = g_2(x) = 1 - x^2
    \]
    and
    \[
      g_1^{-1}(y) = -\sqrt{1-y}, g_2^{-1}(y) = \sqrt{1 - y}
    \]
    Thus
    \begin{align*}
      f_Y(y) &=
      \frac{3}8((\sqrt{1-y} + 1)^2 + (-\sqrt{1-y} +
      1)^2)\left|\frac12(1-y)^{-\frac12}\right| \\
      &=
      \frac{3}8((1-y) + 2\sqrt{1-y} + 1 + (1-y) - 2\sqrt{1-y} +
    1)^2)\left|\frac12(1-y)^{-\frac12}\right| \\
    &=
    \frac{3}8(2(1-y) + 2)\left|\frac12(1-y)^{-\frac12}\right| \\
    &=
    \frac{3}8((1-y) + 1)\left|(1-y)^{-\frac12}\right| \\
    &=
    \frac{3(2-y)}{8(1-y)^{\frac12}}
  \end{align*}
  Further, for $u = 1 - y$
  \begin{align*}
    \int_{0}^1 \frac{3(2-y)}{8(1-y)^{\frac12}} dy &= \frac38 \int
    \frac{2-y}{\sqrt{1-y}} \\
    &= -\frac38 \int \frac{u+1}{\sqrt{u}} \\
    &= -\frac{3}8 \left( \int \sqrt{u} + \int \frac{1}{\sqrt{u}}\right) \\
    &= -\frac{3}8 \left( \frac23(1-y)^{\frac32} + 2\sqrt{1-y} \right) \\
    &= -\frac38\left(-(\frac23 + 2) \right) \\
    &= 1
  \end{align*}
\item
  \[
    f_X(x) = \frac38(x+1)^2 \quad -1 < x < 1 \quad Y = 1 -
    X^2~\text{if}~ X\le0 \text{ and } Y= 1-X \text{ if } X>0
  \]
  As with the previous problem on the partition $A_1 := [-1, 0]$ the
  transformation $g_1(x)= 1 - x^2$ is monotonically increasing, and
  on the partition $A_2 := [0, 1]$ the function is $g_2(x)= 1-x$ the
  function is monotonically decreasing. Finding their inverses:
  \[
    g_1^{-1}(y) = -\sqrt{1-y} \quad g_2^{-1}(y) = 1-x
  \]
  Thus
  \begin{align*}
    f_Y(y) &= \frac{3}8((-\sqrt{(1-y)} +
    1)^2(\frac12(1-y)^{-\frac12}) + (2-y)^2) \\
    &=\frac{3}{8}\left((2-y)^2-1+\frac{2-y}{2\sqrt{1-y}}\right)
  \end{align*}
  Further
  \begin{align*}
    \frac{3}{8}\int_0^1 \left((2-y)^2-1+\frac{2-y}{2\sqrt{1-y}}\right) dy &=
    \frac{3}{8}\left(\int (2-y)^2 - 1 dy +
    \frac12\int\frac{2-y}{2\sqrt{1-y}}\right) \\
    &=
    \frac38\left(\int y^2 - 4y + 5 dy + \frac12\right) \\
    &=
    \frac38\left(\int y^2 - 4y + 5 dy + \frac43\right) \\
    &=
    \frac38\left(\frac43 + \frac43\right)\\
    &= 1
  \end{align*}
\end{enumerate}
\question[2]{}
\begin{enumerate}[(a)]
\item The transformation $Y = X^2$ is monotonically decreasing on the
  interval $A_1 := [-1, 0]$ and monotonically increasingon the
  interval $A_2 := [0, 2]$, however, these intervals fail the third
  conditions of theorem 2.1.8, since $\mathcal{Y}_1 = [0, 1]$ and
  $\mathcal{Y}_2 = [0, 4]$. Clearly the support is $\mathcal{Y} =
  [0,4]$. We can still use the naive method, over the range $[-1,1]$
  \begin{align*}
    P(Y \le y) &= P(X^2 \le y) \\
    &= P(-\sqrt{y} \le X \le \sqrt{y}) \\
    &= \int_{-\sqrt{y}}^{\sqrt{y}} f_X(x) dx \\
    &= \frac29\int_{-\sqrt{y}}^{\sqrt{y}}  x+1 dx \\
    &= \frac{2}{9}\,(x^2+x)\Big|_{x=-\sqrt{y}}^{x=\sqrt{y}} \\
    &= \frac{4\sqrt{y}}9
  \end{align*}
  Over the range $[1,2]$
  \begin{align*}
    P(Y \le y) &= P(X^2 \le y) \\
    &= P(1 \le X \le \sqrt{y})\\
    &= \int_1^{\sqrt{y}} f_X(x) dx \\
    &= \frac{2}{9}\,(x^2+x) \\
    &= \frac{2}{9}(y+{\sqrt{y}} - 2)
  \end{align*}
  Thus after differentiation, the function becomes:
  \[
    f_Y(y) =
    \begin{cases}
      \frac{2}{9\sqrt{y}} \quad x \in [-1,1] \\
      \frac{1}{9}(1+y^{-\frac{1}{2}}) \quad x \in [1, 2]
    \end{cases}
  \]
\item
  For
  \[
    \mathcal{X} \subseteq \bigcup A_i
  \]
  and $g_1...g_k$ which are monotone on each $A_i$ respectively. Then
  by defintion:
  \begin{align*}
    F_Y(y) = P(g(X) \le y) &= \int_{x:g(x)\le y} f_X(x)dx
  \end{align*}
  but because $f_X(x) = 0~\forall x\not\in\mathcal{X}$, the above
  integral is the same as
  \[
    F_Y(y) = \int_{\{x:(g(x) \le y) \} \cap \mathcal{X}} f_X(x)dx
  \]
  The set being integrated over has the partitions:
  \[
    \{x:(g(x) \le y)\} \cap \mathcal{X} =
    \left(\bigcup_{i=1}^{k}\left\{x: g_i(x) \le
    y\right\}\right) \cap \mathcal{X}
  \]
  since each $g_i$ is defined over disjoint $A_i$, the integral can
  be split into
  \[
    F_Y(y) = \sum_{i=1}^k\int_{\{x:(g_i(x) \le y) \} \cap \mathcal{X}} f_X(x)dx
  \]
  Since $g_i$ are all monotone, each interval is either $(-\infty,
  g_i^{-1}(y)]$ if the function is increasing, or $[g_i^{-1}(y),
  \infty)$ when the function is decreasing, so by the fundamental
  theorem of calculus and linearity of derivative:
  \[
    \frac{dF_Y(y)}{dy} = \sum_{i=1}^k f_X(g^{-1}(y))\frac{dg^{-1}{y}}{dx}
  \]
  as desired. When applied to the problem (a): we can choose the
  partitions $A_1 = [-2, 0]$ and $A_2 = [0, 2]$, thus the equations become:
  \[
    g_1^{-1}(y) = - \sqrt{y} \quad g_2^{-1}(y) = \sqrt{y}
  \]
  And so, over the range $y \in [0, 1)$
  \[
    f_Y(y) = \frac{1}{9}y^{-\frac{1}{2}}(\sqrt{y} + 1 - \sqrt{y} + 1)
    = \frac{2}{9}y^{-\frac{1}{2}}
  \]
  but in $[-2, -1]$, $f_X(x) = 0$, so for $y \in [1, 4)$
  \[
    f_Y(y) = \frac{1}{9}y^{-\frac{1}{2}}(\sqrt{y} + 1 )  =
    \frac{1}{9}(1 + y^{-\frac12})
  \]
  Exactly as in part (a).
\end{enumerate}
\end{document}
