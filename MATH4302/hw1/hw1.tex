\documentclass[11pt,letterpaper]{article}
\usepackage[lmargin=1in,rmargin=1in,tmargin=1in,bmargin=1in]{geometry}

% -------------------
% Packages
% -------------------
\usepackage{
  amsmath,      % Math Environments
  amssymb,      % Extended Symbols
  enumerate,        % Enumerate Environments
  graphicx,      % Include Images
  lastpage,      % Reference Lastpage
  multicol,      % Use Multi-columns
  multirow      % Use Multi-rows
}

% -------------------
% Font
% -------------------
\usepackage[T1]{fontenc}
\usepackage{charter}

% -------------------
% Commands
% -------------------
\newcommand{\prob}{\noindent\textbf{Problem. }}
\newcounter{problem}
\newcommand{\problem}{
  \stepcounter{problem}%
  \noindent \textbf{Problem \theproblem. }%
}
\newcommand{\pointproblem}[1]{
  \stepcounter{problem}%
  \noindent \textbf{Problem \theproblem.} (#1 points)\,%
}
\newcommand{\pspace}{\par\vspace{\baselineskip}}
\newcommand{\ds}{\displaystyle}

% -------------------
% Header & Footer
% -------------------
\usepackage{fancyhdr}

\fancypagestyle{doc}{
  %Headers
  \fancyhead[L]{}
  \fancyhead[C]{}
  \fancyhead[R]{}
  \renewcommand{\headrulewidth}{0pt}
  %Footers
  \fancyfoot[L]{}
  \fancyfoot[C]{}
  \renewcommand{\footrulewidth}{0.0pt}
  \headheight=14pt
  \footskip=14pt
}
\pagestyle{doc}

% First Page Style
\newcommand{\homework}[1]{
  \fancypagestyle{first}{
    %Headers
    \fancyhead[L]{\large\bfseries Name: N. Ohayon Rozanes}
    \fancyhead[C]{\bfseries MATH 4302}
    \fancyhead[R]{\bfseries HW #1}
    \renewcommand{\headrulewidth}{1pt}
    %Footers
    \fancyfoot[L]{}
    \fancyfoot[C]{}
    \renewcommand{\footrulewidth}{0.0pt}
  }
  \thispagestyle{first}
}

% -------------------
% Content
% -------------------
\begin{document}
\homework{\#}

% Question 2
\problem

\begin{enumerate}[(a)]
  \item $$\int_0^\infty e^{-ax} dx$$
    For $f(x) := e^{-ax}$, which has a domain over the entire real
    number line, we have that the improper integral is equal to
    \[
      \int_0^\infty e^{-ax} = \lim_{b\to\infty}\int_0^b e^{-ax}
    \]
    The integral on the right hand side evaluates to:
    \begin{align*}
      \lim_{b\to\infty}\int_0^b e^{-ax} &= \lim_{b\to\infty}
      \frac{-1}{a}e^{-ax}\Big|_0^b \\
      &= \lim_{b\to\infty}\left(\frac{-1}{a}(e^{-ab} - 1)\right) \\
      &= \left(\frac{-1}{a}(-1)\right) \\
      &= \frac{1}{a}
    \end{align*}
  \item
    Define:
    \[
      I(a,b)\;:=\;\int_{0}^{\infty} e^{-ax}\cos(bx)\,dx,\qquad a>0.
    \]
    Let
    \[
      u=\cos(bx),\qquad dv=e^{-ax}\,dx.
    \]
    Then
    \[
      du=-b\sin(bx)\,dx,\qquad v=-\frac1a e^{-ax}.
    \]
    Using integration by parts
    \begin{align*}
      I(a,b)
      &=\int_{0}^{\infty} u\,dv \\
      &=\Big[u\,v\Big]_{0}^{\infty}-\int_{0}^{\infty} v\,du \\
      &=\Big[-\frac1a
      e^{-ax}\cos(bx)\Big]_{0}^{\infty}-\int_{0}^{\infty}\Big(-\frac1a
      e^{-ax}\Big)\big(-b\sin(bx)\big)\,dx \\
      &=\Big[-\frac1a
      e^{-ax}\cos(bx)\Big]_{0}^{\infty}-\frac{b}{a}\int_{0}^{\infty}
      e^{-ax}\sin(bx)\,dx.
    \end{align*}
    Evaluating the first term:
    \[
      \lim_{x\to\infty}\left(-\frac1a e^{-ax}\cos(bx)\right)=0
      \quad\text{(since \(a>0\))},\qquad
      -\frac1a e^{-a\cdot 0}\cos(0)=-\frac1a.
    \]
    Hence
    \[
      \Big[-\frac1a e^{-ax}\cos(bx)\Big]_{0}^{\infty}=0-(-\tfrac1a)=\frac1a,
    \]
    and therefore
    \[
      I(a,b)=\frac1a-\frac{b}{a}J(a,b),
      \qquad
      J(a,b):=\int_{0}^{\infty} e^{-ax}\sin(bx)\,dx.
    \]

    Compute \(J(a,b)\) by parts with
    \[
      u=\sin(bx),\qquad dv=e^{-ax}\,dx,
    \]
    so
    \[
      du=b\cos(bx)\,dx,\qquad v=-\frac1a e^{-ax}.
    \]
    Then
    \begin{align*}
      J(a,b)
      &=\int_{0}^{\infty} u\,dv \\
      &=\Big[u\,v\Big]_{0}^{\infty}-\int_{0}^{\infty} v\,du \\
      &=\Big[-\frac1a
      e^{-ax}\sin(bx)\Big]_{0}^{\infty}-\int_{0}^{\infty}\Big(-\frac1a
      e^{-ax}\Big)\big(b\cos(bx)\big)\,dx \\
      &=\Big[-\frac1a
      e^{-ax}\sin(bx)\Big]_{0}^{\infty}+\frac{b}{a}\int_{0}^{\infty}
      e^{-ax}\cos(bx)\,dx \\
      &=\Big[-\frac1a e^{-ax}\sin(bx)\Big]_{0}^{\infty}+\frac{b}{a}I(a,b).
    \end{align*}
    Evaluate the boundary term:
    \[
      \lim_{x\to\infty}\left(-\frac1a e^{-ax}\sin(bx)\right)=0,\qquad
      -\frac1a e^{-a\cdot 0}\sin(0)=0,
    \]
    so the boundary contribution is \(0-0=0\). Hence
    \[
      J(a,b)=\frac{b}{a}I(a,b).
    \]

    \bigskip
    \[
      I(a,b)=\frac1a-\frac{b}{a}J(a,b).
    \]
    Insert \(J(a,b)=\frac{b}{a}I(a,b)\):
    \begin{align*}
      I(a,b)
      &=\frac1a-\frac{b}{a}\left(\frac{b}{a}I(a,b)\right) \\
      &=\frac1a-\frac{b^{2}}{a^{2}}I(a,b).
    \end{align*}
    Bring the \(I\)-terms together:
    \begin{align*}
      I(a,b)+\frac{b^{2}}{a^{2}}I(a,b)
      &=\frac1a \\
      I(a,b)\left(1+\frac{b^{2}}{a^{2}}\right)
      &=\frac1a \\
      I(a,b)\left(\frac{a^{2}+b^{2}}{a^{2}}\right)
      &=\frac1a.
    \end{align*}
    Therefore
    \[
      I(a,b)=\frac1a\cdot \frac{a^{2}}{a^{2}+b^{2}}
      =\frac{a}{a^{2}+b^{2}}.
    \]

    \bigskip

    \[
      \boxed{\displaystyle \int_{0}^{\infty}
      e^{-ax}\cos(bx)\,dx=\frac{a}{a^{2}+b^{2}}\qquad (a>0).}
    \]

\end{enumerate} \vspace{6cm}

\vfill
\problem
\[
  \int_0^\infty \frac{\sin x}{x}dx = \frac\pi2
\]
\begin{enumerate}[(a)]
  \item
    \begin{align*}
      \int_0^\infty \frac{\sin{ax}}{x}dx  &= I \\
      \frac1a\int_0^\infty \frac{\sin{ax}}{x}dx  &= \frac Ia \\
      \int_0^\infty \frac{\sin{ax}}{ax}dx  &= \frac Ia \\
    \end{align*}
    let $u = ax$ then $du = a dx$, also at $x=0 \implies u = 0$ and
    as $x\to\infty \implies u \to \infty$ thus
    \begin{align*}
      \frac{1}a\int_0^\infty \frac{\sin{ax}}{ax}adx  &= \frac Ia \\
      \frac{1}a\int_0^\infty\frac{\sin{ax}}{ax}adx  &= \frac Ia \\
      \int_0^\infty\frac{\sin{u}}{u}du  &= I \\
      \frac{\pi}2  &= I
    \end{align*}
  \item using the identity that $\sin\alpha\cos\beta =
  \frac12(\sin(\alpha + \beta)) + \sin(\alpha - \beta))$, we can
  rewrite the integrand as
  \[
    \frac{\sin{ax}\cos{bx}}{x} = \frac{\sin(x(a+b)) + \sin(x(a-b))}{2x}
  \]
  Thus the integral becomes:
  \begin{align*}
    \int_0^\infty \frac{\sin{ax}\cos{bx}}{x} &=
    \int_0^\infty\frac{\sin(x(a+b)) + \sin(x(a-b))}{2x} \\
    &=
    \int_0^\infty\frac{\sin(x(a+b))}{2x} + \int^\infty_0
    \frac{\sin(x(a-b))}{2x} \\
    &=
    \frac12\left(\int_0^\infty\frac{\sin(x(a+b))}{x} + \int^\infty_0
    \frac{\sin(x(a-b))}{x}\right)
  \end{align*}
  Define $u_1 = a + b$ and $u_2 = a - b$
  \[
    \frac12\left(\int_0^\infty\frac{\sin(xu_1)}{x} + \int^\infty_0
    \frac{\sin(xu_2)}{x}\right)
  \]
  Apply the result from part (a)
  \[
    \frac12\left(\int_0^\infty\frac{\sin(xu_1)}{x} + \int^\infty_0
    \frac{\sin(xu_2)}{x}\right) = \frac{1}2\left(\frac\pi2 +
    \frac\pi2\right) = \frac\pi2
  \]
\end{enumerate} \vspace{6cm}
\problem
\begin{enumerate}[(a)]
\item
  \[
    \int_2^\infty \frac{dx}{x^k\ln x} = \int_2^\infty \frac{x^{-k}}{\ln x} dx
  \]
  Notice that $\frac{1}{\ln x} \le 1$ is satisified for $x \ge e$, thus
  for $f(x) := \frac{x^{-k}}{\ln x}$ and $g(x) := x^{-k}$, then
  \[
    0 \le f(x) \le g(x)
  \]
  $\forall x \ge e \ge 2$. Using the comparison test, if we can find
  the values of $k$ for which
  $\int_2^\infty g(x) dx$ converges, then $\int_2^\infty f(x) dx$ will
  converge for those values.

  It is known from lecture that $\int_1^\infty g(x) dx$
  will diverge for any $k\le1$, a similar proof can be repeated for
  $\int_2^\infty g(x) dx$: assume that $k > 1$, then
  \begin{align*}
    \int_2^\infty x^{-k} &= \lim_{b\to\infty}\int_2^b x^{-k}  =
    \lim_{b\to\infty} \frac{1}{1 - k}x^{-k + 1}\Big|_2^b \\
    &= -\frac{2^{-k+1}}{1-k}
  \end{align*}
  If $k = 1$
  \begin{align*}
    \int_2^\infty x^{-1} = \lim_{b\to\infty}\int_2^b x^{-1} dx =
    \lim_{b\to\infty} \ln b - \ln 2 = \infty
  \end{align*}
  Finally if $k < 1$:
  \begin{align*}
    \int_2^\infty x^{-k} &= \lim_{b\to\infty}\int_2^b x^{-k}  =
    \lim_{b\to\infty} \frac{1}{1 - k}x^{-k + 1}\Big|_2^b \\
    &= +\infty
  \end{align*}
  Therefore, $\int_2^\infty f(x)dx$  will converge for $k < 1$.
\item Since $\forall x \in [1, \infty) : e^{-x} \l 1 \implies
  x^pe^{-x} < x^p$ then the integral:
  \[
    \int_1^\infty x^pe^{-x} dx
  \]
  since $\int_1^\infty x^p dx$ converges for any $p < -1$, then by
  the comparison test so too does $\int_1^\infty x^pe^{-x} dx$. For
  $p \ge -1$ we can use successive integration by parts, set
  \[
    u = x^p, dv = e^{-x} dx \implies du = px^{p-1}dx, v = -e^{-x}
  \]
  \[
    \int_1^\infty x^pe^{-x} dx = -e^{-x}x^p + p\int_1^\infty x^{p-1}e^{-x}dx
  \]
  The integration can be repeated $n$ times, where $p - n < - 1$,
  which results in the expression:
  \[
    \begin{aligned}
      I
      &= -e^{-x}x^p
      + p\Big(
        -e^{-x}x^{p-1}
        + (p-1)\Big(
          -e^{-x}x^{p-2}
          + \cdots \\
          &\hspace{3cm}
          + (p-n+1)\int_1^\infty x^{p-n}e^{-x}\,dx
        \Big)
      \Big)
    \end{aligned}
  \]

  The last term converges to some $M$, as we've shown. Therefore it
  suffices to show that the leading terms are also finite, that is

  \[
    \begin{aligned}
      &-e^{-x}x^p
      + p\Big(
        -e^{-x}x^{p-1}
        + (p-1)\Big(
          -e^{-x}x^{p-2}
          + \cdots \\
          &\hspace{3cm}
          + (p-n)\Big(
            -e^{-x}x^{p-n}
            + (p-n+1)M
          \Big)
        \Big)
      \Big)
      \;\le\; \infty
    \end{aligned}
  \]

  For any \(k\)
  \[
    e^{-x}x^k\Big|_1^\infty = 0 - e^{-1}
  \]

  so the above sum becomes:
  \[
    \begin{aligned}
      e^{-1}
      + p\Big(
        e^{-1}
        + (p-1)\Big(
          e^{-1}
          + \cdots
          + (p-n)\Big(
            e^{-1}
            + (p-n+1)M
          \Big)
        \Big)
      \Big)
    \end{aligned}
  \]
  After destributing, the sum clearly becomes:
  \[
    e^{-1}\sum_0^n \frac{p!}{(p-n)!} + \left(\frac{p!}{(p-n)!}\right)M
  \]
  Which is clearly finite. Thus, the integral converges for all $p$
\end{enumerate}
\problem
\begin{enumerate}[(a)]
  %% Goal: make the contradiction proof “honestly improper-integral” by
  %% writing everything in terms of the positive/negative parts f^+, f^-.

\item
  Assume $\displaystyle \int_a^\infty f(x)\,dx$ converges (as an
  improper integral)
  and that $\lim_{x\to\infty} f(x)=\alpha$ exists. We show $\alpha=0$
  by contradiction.

  Recall the positive and negative parts:
  \[
    f^+(x):=\max\{f(x),0\},\qquad f^-(x):=\max\{-f(x),0\},
  \]
  so that
  \[
    f(x)=f^+(x)-f^-(x),\qquad |f(x)|=f^+(x)+f^-(x).
  \]

  \medskip
  Assume that $\alpha > 0$ and set $\varepsilon:=\alpha/2$. Then
  $\exists\,M$ such that for all $x>M$,
  \[
    |f(x)-\alpha|<\frac{\alpha}{2}
    \quad\Longrightarrow\quad
    \frac{\alpha}{2}<f(x)<\frac{3\alpha}{2}.
  \]
  In particular, for $x>M$ we have $f(x)>0$, hence
  \[
    f^+(x)=f(x)\ge \frac{\alpha}{2},
    \qquad
    f^-(x)=0
    \qquad (x>M).
  \]
  Define $g(x):=\alpha/2$. Then for $x>M$,
  \[
    0<g(x)\le f^+(x).
  \]
  Now compute:
  \[
    \int_M^\infty g(x)\,dx
    =\lim_{b\to\infty}\int_M^b \frac{\alpha}{2}\,dx
    =\lim_{b\to\infty}\frac{\alpha}{2}(b-M)
    =+\infty.
  \]
  By comparison on $[M,\infty)$,
  \[
    \int_M^\infty f^+(x)\,dx=+\infty.
  \]
  But since $f^-(x)\ge 0$, for every $b>M$,
  \[
    \int_M^b f(x)\,dx=\int_M^b f^+(x)\,dx-\int_M^b f^-(x)\,dx
    \ge \int_M^b f^+(x)\,dx-\int_M^b 0\,dx
    =\int_M^b f^+(x)\,dx.
  \]
  Letting $b\to\infty$ forces $\int_M^\infty f(x)\,dx=+\infty$, contradicting
  convergence. Hence $\alpha\not>0$.

  \medskip
  Now assume $\alpha < 0$ and set $\varepsilon:=-\alpha/2>0$. Then
  $\exists\,M$ such that for all $x>M$,
  \[
    |f(x)-\alpha|<-\frac{\alpha}{2}
    \quad\Longrightarrow\quad
    \frac{3\alpha}{2}<f(x)<\frac{\alpha}{2}<0.
  \]
  So for $x>M$ we have $f(x)<0$, hence
  \[
    f^+(x)=0,\qquad
    f^-(x)=-f(x)\ge -\frac{\alpha}{2}
    \qquad (x>M).
  \]
  Define $h(x):=-\alpha/2>0$. Then for $x>M$,
  \[
    0<h(x)\le f^-(x).
  \]
  Compute:
  \[
    \int_M^\infty h(x)\,dx
    =\lim_{b\to\infty}\int_M^b \left(-\frac{\alpha}{2}\right)\,dx
    =\lim_{b\to\infty}\left(-\frac{\alpha}{2}\right)(b-M)
    =+\infty,
  \]
  so by comparison,
  \[
    \int_M^\infty f^-(x)\,dx=+\infty.
  \]
  But then for every $b>M$,
  \[
    \int_M^b f(x)\,dx
    =\int_M^b 0\,dx-\int_M^b f^-(x)\,dx
    =-\int_M^b f^-(x)\,dx,
  \]
  and letting $b\to\infty$ gives $\int_M^\infty f(x)\,dx=-\infty$,
  again contradicting convergence. Hence $\alpha\not<0$.
  \medskip
  Therefore $\alpha=0$.
\item Using the FTC
  \[
    \int_a^\infty f'(x)dx = \lim_{b\to\infty}\int_a^b f'(x)dx =
    \lim_{b\to\infty}f(b) - f(a)
  \]
  By assumption $\int_a^\infty f'(x)dx$ converges, so the limit
  converges to some finite $\alpha$:
  \[
    \lim_{b\to\infty}f(b) = \alpha
  \]
  From part (a), we know that $\alpha = 0$ then:
  \[
    \lim_{x\to\infty}f(x) = 0
  \]
\end{enumerate}

\problem
\begin{enumerate}[(a)]
\item
  \[
    \int_0^a x^pe^{1/x}dx
  \]
  Notice that $\forall x \in (0, \infty)$
  \[
    e^{1/x} > 1
  \]
  Therefore $x^pe^{1/x} > x^p,~\forall x \in (0, a]$. For $p \le -1$,
  integrating $x^p$ directly yields:
  \[
    \int_0^a x^p dx = \lim_{\varepsilon\to0^+}
    \frac{x^{p+1}}{p+1}\Big|_\varepsilon^a
  \]
  At $p = -1$, the  limit diverges because of $\frac{1}{p+1}$, for
  $p < -1$ as  $x\to0,~x^{p+1} \to \infty$, so the limit diverges. By
  the comparison test,
  \[
    \int_0^a e^{1/x}x^p dx
  \]
  diverges for $p \le -1$. Assuming that $p > -1$, by the archimedian
  property, there exists an integer such that $n > p + 1 \implies p -
  n < -1$. Further, recall that by Taylor's theorem:
  \[
    e^{\frac1x} = \sum_{k=0}^\infty \frac{1}{k!x^k} > \frac{1}{n!x^n}
  \]
  for any $n \in \mathbb{N}$. Therefore:
  \[
    x^pe^{1/x} > \frac{x^{p-n}}{n!}
  \]
  Further, since $p - n < -1$, and we've already shown that the integral
  \[
    \frac{1}{n!}\int_0^a x^{p-n} dx
  \]
  diverges, then by the comparison test, the integral
  \[
    \int_0^a x^pe^{1/x} dx
  \]
  also diverges for $p > -1$, and thus diverges for all $p$.
\item
  \[
    \int_0^a \ln x dx
  \]
  This integral is improper since as $x\to0, \ln x \to \infty$.
  The proper integral can be evaluated directly by integration by
  parts, let $u = \ln x$ and $dv = 1$, $du = 1/x$ and $v = x$
  \begin{align*}
    \int_\varepsilon^a \ln x dx &= x\ln x - \int 1 dx\\
    &= x\ln x - x\Big|_\varepsilon^a \\
    &= (a \ln a - a) - (\varepsilon \ln \varepsilon - \varepsilon)
  \end{align*}

  Evaluating the limit as $\varepsilon \to 0$:
  $$
  \lim_{\varepsilon\to0^+} \varepsilon\ln\varepsilon =
  \lim_{\varepsilon\to0^+} \frac{\ln\varepsilon}{\frac1\varepsilon}
  $$
  Which by L'Hospital's theorem:
  \[
    \lim_{\varepsilon\to0^+} \frac{\ln\varepsilon}{\frac1\varepsilon} =
    \lim_{\varepsilon\to0^+}
    \frac{\frac{1}\varepsilon}{\frac{-1}{\varepsilon^2}} =
    \lim_{\varepsilon\to0^+} -\varepsilon = 0
  \]
  Thus, the integral becomes:
  \[
    \lim_{\varepsilon\to0^+} \int_\varepsilon^a \ln x dx = a \ln a - a
  \]
  For all $a$.
\item The integral is improper on both ends, as $x \to 1$
  $\frac{1}{\ln x}\to \infty$. First, notice who, for $x\in (1,
  \infty)$ $\frac{1}{x\ln x} < \frac{1}{\ln x}$, thus by the
  comparison test, if we can show that the integral
  \[
    \int_1^\infty \frac{1}{x\ln x} dx
  \]
  diverges, so too does the original integral. The Integral above can
  be evaluated directly through $u$ substitution, let $u = \ln x, du
  =\frac1x dx$
  \begin{align*}
    \lim_{(\varepsilon, b)\to(0,\infty)}\int_{1+\varepsilon}^b
    \frac{1}{x\ln x} dx &= \int \frac{1}{u} du \\
    &= \ln u \\
    &= \lim_{(\varepsilon, b) \to (0,
    \infty)}\ln(\ln(x))\Big|_{1+\varepsilon}^b \\
    &= \lim_{(\varepsilon, b) \to (0, \infty)}\ln\ln(b) - \ln\ln(1+\epsilon)
  \end{align*}
  Both terms tend to infinity as $(\varepsilon, b) \to (0, \infty)$,
  clearly the integral diverges, and thus so too does the original integral.
\end{enumerate}

\end{document}
