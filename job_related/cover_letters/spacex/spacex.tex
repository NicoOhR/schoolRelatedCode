% Cover letter using letter.cls
\documentclass{letter}
%\usepackage{helvetica} % uses helvetica postscript font (download
% helvetica.sty)
%\usepackage{newcent}   % uses new century schoolbook postscript font
% the following commands control the margins:
\topmargin=-1in    % Make letterhead start about 1 inch from top of page
\textheight=10in    % text height can be bigger for a longer letter
\oddsidemargin=0pt % leftmargin is 1 inch
\textwidth=6.5in   % textwidth of 6.5in leaves 1 inch for right margin

\begin{document}

\signature{N. Ohayon Rozanes}                  % name for signature
\longindentation=0pt                     % needed to get closing flush left
\let\raggedleft\raggedright              % needed to get date flush left

\begin{letter}{}

  \begin{center}
    {\large\bf N. Ohayon Rozanes} \\
    {611 Granbury Dr. \\ Allen, Texas, 75013  \\ 469-560-0141}
  \end{center} \vfill % forces letterhead to top of page

  \opening{Dear Hiring Manager:}

  \noindent I am writing to apply for the Software Engineering
  Internship for Spring 2026 at SpaceX. I was reccomended to apply
  here by Will Ferenec when I interned at Exolambda. I am pursuing
  double degree Data Science and Mathematics at the university of
  Texas at Dallas, and on a
  technical level I specialize in Embedded firmware development.
  While the connection between mathematics and firmware seems tenuous
  at first, my interest in either stems from the same basic interest;
  I love working from first principles. To that end, I am interested
  in roles that involve writing and validating real-time software
  that operates close to the hardware and which has visceral,
  physical impact on the end product. SpaceX's emphasis on autonomous
  spacecraft systems and reliable flight software aligns with both my
  experience and long term goals. To elaborate a little bit on the
  last point, maybe a year ago I became determined to only write
  software that, at least to some extent, matters. In the case of
  personal projects this is easy enough since there is always some
  pedagogical or aesthetic value to those. Professionally, it often feels
  like a lot of software (not naming any names) is basically
  meaningless, working towards
  ends that I personally do not value. In that way, SpaceX's mission
  to work towards the betterment of humanity, to engineer our way
  into a future worth living in, is something that I am very
  compelled to build software.

  \noindent I have substantial experience developing software for
  embedded targets, both in a professional and amateur environment,
  as well as in a team oriented and independent project contexts. My
  work frequently involved C and C++ development, reading and
  applying datasheets and protocol specs, and debugging failures that
  span the hardware/software boundary. I am most at home working in
  Linux environments. In addition to coursework in numerical
  analysis, computer architecture and data structures, I continuously
  pursue hands-on projects outside of the classroom to sharpen my
  abilities in these domains. At the UTD FSAE club, Dallas Formula
  Racing, I headed the embedded team for about a year before stepping
  down, but I still contribute to firmware development, PCB design,
  and composites manufacturing. I am particularly drawn to flight
  software because of the demands made on software: interfaces must
  be explicit, failure modes well understood, and even in
  extraordinary circumstances, the software must continue to function
  correctly, which pushes me to engineer better solutions. Also, I really
  enjoyed working with people in
  the necessarily cross-disciplinary aerospace industry during my
  internship at Exolambda. I have attached to the application my
  portfolio in pdf form, but for a better experience I reccomend my
  website at: https://nicoohr.github.io/Sicarii.

  \noindent SpaceX's mission is compelling because it treats space
  flight as something that can be made practical, more of an
  engineering problem to be solved and less hypothetical platitudes.
  The opportunity to contribute to software that directly enables
  guidance, navigation, control, power, and communications on a real
  vehicle is exactly the kind of responsibility I am seeking in an
  internship. Please let me know if there is anything else I can
  provide you with during this process.

  \closing{Sincerely yours,}
\end{letter}

\end{document}
