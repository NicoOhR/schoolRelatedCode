% Cover letter using letter.cls
\documentclass{letter}
%\usepackage{helvetica} % uses helvetica postscript font (download
% helvetica.sty)
%\usepackage{newcent}   % uses new century schoolbook postscript font
% the following commands control the margins:
\topmargin=-1in    % Make letterhead start about 1 inch from top of page
\textheight=10in    % text height can be bigger for a longer letter
\oddsidemargin=0pt % leftmargin is 1 inch
\textwidth=6.5in   % textwidth of 6.5in leaves 1 inch for right margin

\begin{document}

\signature{N. Ohayon Rozanes}                  % name for signature
\longindentation=0pt                     % needed to get closing flush left
\let\raggedleft\raggedright              % needed to get date flush left

\begin{letter}{}

  \begin{center}
    {\large\bf N. Ohayon Rozanes} \\
    {611 Granbury Dr. \\ Allen, Texas, 75013  \\ 469-560-0141}
  \end{center} \vfill % forces letterhead to top of page

  \opening{Dear Hiring Manager:}

  \noindent
  I am applying for the Software Developer Intern role at Amperesand.
  I am currently pursuing a double major in Data Science and
  Mathematics, with a strong focus on systems programming, real-time
  software, and hardware–software integration. Amperesand’s work on
  solid-state transformer platforms directly aligns with my interests
  in embedded control, power-adjacent systems, and software that must
  perform correctly under physical and timing constraints.

  \noindent
  I have substantial experience developing software for
  embedded targets, both in a professional and amateur environment,
  as well as in a team oriented and independent project contexts. My
  work frequently involved C and C++ development, reading and
  applying datasheets and protocol specs, and debugging failures that
  span the hardware/software boundary. I am most at home working in
  Linux environments. In addition to coursework in numerical
  analysis, computer architecture and data structures, I continuously
  pursue hands-on projects outside of the classroom to sharpen my
  abilities in these domains. At the UTD FSAE club, Dallas Formula
  Racing, I headed the embedded team for about a year before stepping
  down, but I still contribute to firmware development, PCB design,
  and composites manufacturing. I am particularly drawn to flight
  software because of the demands made on software: interfaces must
  be explicit, failure modes well understood, and even in
  extraordinary circumstances, the software must continue to function
  correctly, which pushes me to engineer better solutions. Also, I really
  enjoyed working with people in
  the necessarily cross-disciplinary start up during my
  internship at Exolambda.

  \noindent I am comfortable working in lab  environments,
  validating software against real hardware, and iterating quickly
  when theory meets physical reality. I value clear interface
  definitions, disciplined architecture, and testing that reflects
  real operating conditions rather than idealized assumptions. I also
  actively contribute to and maintain personal and academic projects,
  several of which are open-source and focused on systems-level
  correctness and performance.

  \noindent Amperesand’s mission to rethink grid infrastructure
  through solid-state power platforms is technically ambitious and
  operationally grounded. I am motivated by environments where
  software quality directly impacts safety, reliability, and system
  performance, and where engineers are expected to take ownership
  from design through deployment.
  I would welcome the opportunity to contribute to Amperesand’s
  embedded and edge software efforts and to learn from a team
  operating at the intersection of power electronics and real-time systems.

  \closing{Sincerely yours,}
\end{letter}

\end{document}
