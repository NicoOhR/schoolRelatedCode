%-------------------------
% Rover Resume - Base Template
% Link: https://github.com/subidit/rover-resume
%
% Shows code for various formatting options for different resume sections.
% Education and Projects have single-line headers; while Experience uses double-line.
% Some formatting codes are kept inline; consider \newcommand{cmd}{def}.
% Excludes hyperref and icons for readability; MVP version.
% Explore other templates for more options.
% Mix and match as desired. Be consistent with headers and sub-headers.
%------------------------

\documentclass[11pt]{article} % fontsize 10pt/11pt/12pt

\usepackage[margin=1in, a4paper]{geometry}
\setcounter{secnumdepth}{0} % remove section numbering
\usepackage{titlesec}
\titlespacing{\subsection}{0pt}{*0}{*0} % remove vertical spacing above and below
\titlespacing{\subsubsection}{0pt}{*0}{*0}
\titleformat{\section}{\large\bfseries\uppercase}{}{}{}[\titlerule]
\titleformat*{\subsubsection}{\large\itshape}
\usepackage{enumitem}
\setlist[itemize]{noitemsep,left=0pt .. \parindent}
\usepackage[colorlinks=true,urlcolor=blue,linkcolor=blue,citecolor=blue]{hyperref}
\usepackage{url}  
\urlstyle{same}    
\pagestyle{empty} % remove page number

\begin{document}

\begin{center}
	\begin{minipage}{0.4\textwidth}
		{\Huge\bfseries
			Nico Ohayon
		} \\ \medskip
	\end{minipage} \hfill
	\begin{minipage}{0.55\textwidth}
				\begin{tabular*}{\textwidth}{@{}l@{\extracolsep{\fill}}r@{}}
            Email:    & nohayonr@gmail.com \\
            Mobile:   & 469 560 0141 \\
            LinkedIn: & linkedin.com/in/nico-ohayon-rozanes/ \\
            GitHub:   & github.com/NicoOhR \\
            Portfolio:& nicoohr.github.io/Sicarii/index.html \\
        \end{tabular*}
	\end{minipage}
\end{center}

\section{Education}
%=================%
\subsection{University of Texas at Dallas, Richardson $|$ {\normalfont\itshape Mathematics (B.S.) and Data Science (B.S.)} \hfill Graduating May 2027}
\begin{itemize}
	\item Cumulative GPA: 3.45
\end{itemize}
\section{Experience}
%=================%
\subsection{Exolambda, Richardson}
\subsubsection{Software Engineering Intern \hfill  May 2025 - Aug 2025}
\begin{itemize}
    \item Supported the bring-up and validation of custom embedded avionics platforms, including board-level diagnostics, firmware initialization, and integration of sensing and compute modules.
    \item Implemented data acquisition and preprocessing pipelines for onboard IMU, GPS, and environmental sensors, enabling downstream statistical analysis and control-loop tuning.
    \item Developed and profiled real-time control and signal-processing routines, emphasizing numerical stability, latency bounds, and deterministic behavior under flight-critical constraints.
    \item Contributed to the design of internal monitoring and testing procedures used to inspect system performance, analyze failure modes, and quantify reliability metrics across development iterations.
		\item Modernized build system tooling for embedded targets.
    \item Collaborated with cross-disciplinary engineers to prototype algorithms linking sensor fusion, state estimation, and fault detection, with a focus on quantitative validation using collected flight-test data.
\end{itemize}
\section{Research}
%===============================%
\subsection{Ganglia Guardian | Dr. Gu}
\subsubsection{Statistics and Bioinformatics Research \hfill August 2025 - Current}
\begin{itemize}
    \item Collaborate with Dr.\ Gu and graduate researchers on quantitative methods for analyzing rodent behavioral data collected from high-frequency IMU sensors, with the objective of identifying early somatic indicators of neurodegenerative disease, inspired by Tang et al. (2024).
    \item Optimize and profile unsupervised clustering pipelines—primarily Affinity Propagation—focusing on numerical stability, convergence behavior, and large-scale similarity-matrix computation.
    \item Implement GPU-accelerated primitives for similarity computation, distance kernels, and iterative message-passing updates, achieving significant reductions in runtime for high-dimensional behavioral datasets.
    \item Develop preprocessing routines for IMU time-series, including filtering, windowing, and statistical feature extraction, improving downstream cluster separability and interpretability.
    \item Build documentation and experiment-tracking infrastructure enabling reproducible analyses, parameter sweeps, and quantitative comparisons across algorithmic variants.
\end{itemize}
\subsection{Undergraduate Research Scholar Award | Prof. Bereg}
\subsubsection{Mathematics and Computer Science \hfill Fall 2024}
\begin{itemize}
    \item Investigated transversals in Latin squares under the supervision of Prof.\ Bereg, focusing on computational verification of results from Wanless (2009).
    \item Designed and implemented efficient C++ algorithms for large combinatorial matrices, emphasizing pruning strategies, symmetry reduction, and high-performance enumeration.
    \item Conducted numerical experiments to explore conjectured patterns and structural invariants in high-order Latin squares.
\end{itemize}
\subsection{ACM Research Trade X | Prof. Smiley}
\subsubsection{Statistics $|$ https://github.com/ACM-TradeX/TradeX \hfill Spring 2024}
\begin{itemize}
    \item Applied the PANDA algorithm (Glass et al.) to equity time series to identify strongly co-moving assets for potential statistical arbitrage.
    \item Built preprocessing pipelines for financial returns, computing covariance structures and regularized precision matrices for dependency inference.
    \item Evaluated asset-pair selection strategies through simulation and exploratory data analysis to assess robustness across varying market regimes.
\end{itemize}

\section{Projects}
%=================
\subsection{NoTeC and WBSS $|$ \normalfont\textit{github.com/NicoOhR/DataAcquisition2.0}}
\begin{itemize}
	\item Led the design and implementation of NoTeC, a modular embedded data-acquisition system for a student FSAE vehicle, using an STM32 (migrating from F4 to H7), C++ and CMSIS-RTOS, and abstracted hardware via interface layers to decouple sensors and platform. 
	\item Implemented the successor to NoTeC, the Wheel Base Sensor System, based on the esp32 platform. Allowed for fine tuning of DAQ processes by partitioning the system into individual nodes allocated per wheel.
\end{itemize}
\subsection{Three Body Simulation (GV3B) $|$ \normalfont\textit{github.com/NicoOhR/GV3B}}
\begin{itemize}
	\item Built a physics simulation of the classical three-body problem in Rust using Bevy + Rapier2D, including vector visualisations and a custom gravitational solver to explore dynamic stability and escape/collision conditions. 
	\item Exposed the simulation via a gRPC service (using Tonic) so that external clients (e.g., a Python script) can configure initial conditions, query states, and conduct programmatic experiments for numerical research. 
\end{itemize}
\subsection{FSAEStats $|$ \normalfont\textit{github.com/NicoOhR/FSAEStats}}
\begin{itemize}
	\item Automated extraction of competition results from PDF-formatted reports (using Camelot) and transformed them into usable CSV/Apache Arrow formats for statistical modelling of race performance and team trends. 
	\item Built a server backend in Rust using DuckDB + Apache Arrow for flexible, high-performance querying of race event data and exposed endpoints to deliver analysis-ready datasets to users.
\end{itemize}
\section{Skills}
\begin{description}[itemsep=0pt]
	\item[Languages] C/C++, Rust, Python, GoLang, \LaTeX
  \item[Tools]: Linux, BusyBox, Yocto, Kafka, Docker, Git, CMake
  \item[Data/ML]: SQL, NumPy, pandas, scikit-learn, PyTorch, Matplotlib/Seaborn, DuckDB, Apache Arrow
	\item[Language] English, Spanish, Hebrew
\end{description}
\end{document}
