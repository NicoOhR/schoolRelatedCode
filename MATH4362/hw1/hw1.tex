\documentclass[11pt,letterpaper]{article}
\usepackage[lmargin=1in,rmargin=1in,tmargin=1in,bmargin=1in]{geometry}

% -------------------
% Packages
% -------------------
\usepackage{
  amsmath,      % Math Environments
  amssymb,      % Extended Symbols
  enumerate,        % Enumerate Environments
  graphicx,      % Include Images
  lastpage,      % Reference Lastpage
  multicol,      % Use Multi-columns
  multirow      % Use Multi-rows
}

% -------------------
% Font
% -------------------
\usepackage[T1]{fontenc}
\usepackage{charter}

% -------------------
% Commands
% -------------------
\newcommand{\prob}{\noindent\textbf{Problem. }}
\newcounter{problem}
\newcommand{\problem}{
  \stepcounter{problem}%
  \noindent \textbf{Problem \theproblem. }%
}
\newcommand{\pointproblem}[1]{
  \stepcounter{problem}%
  \noindent \textbf{Problem \theproblem.} (#1 points)\,%
}
\newcommand{\pspace}{\par\vspace{\baselineskip}}
\newcommand{\ds}{\displaystyle}

% -------------------
% Header & Footer
% -------------------
\usepackage{fancyhdr}

\fancypagestyle{doc}{
  %Headers
  \fancyhead[L]{}
  \fancyhead[C]{}
  \fancyhead[R]{}
  \renewcommand{\headrulewidth}{0pt}
  %Footers
  \fancyfoot[L]{}
  \fancyfoot[C]{}
  \renewcommand{\footrulewidth}{0.0pt}
  \headheight=14pt
  \footskip=14pt
}
\pagestyle{doc}

% First Page Style
\newcommand{\homework}[1]{
  \fancypagestyle{first}{
    %Headers
    \fancyhead[L]{\large\bfseries Name: Nico Ohayon Rozanes }
    \fancyhead[C]{\bfseries MATH 4362}
    \fancyhead[R]{\bfseries HW #1}
    \renewcommand{\headrulewidth}{1pt}
    %Footers
    \fancyfoot[L]{}
    \fancyfoot[C]{}
    \renewcommand{\footrulewidth}{0.0pt}
  }
  \thispagestyle{first}
}

% -------------------
% Content
% -------------------
\begin{document}
\homework{\#}

% Question 1
\problem \vspace{0.25cm}

I decided to take this course a bit earlier than I would have
otherwise because I have recently taken up studying optimal
transport, which has application in the analysis of certain classes of PDEs.

% Question 2
\problem Exercise 1.27\vspace{0.25cm}

\begin{enumerate}[(a)]
  \item $$u' - 4u = x - 3$$
    Assume the existence of a function $\mu(x)$ such that $-4\mu(x) =
    \mu'(x)$ and multipy the equation through.
    \begin{align*}
      u'\mu(x) - 4u\mu(x) &= \mu(x)(x-3) \\
      u'\mu(x) + \mu'(x) &= \mu(x)(x-3) \\
      (u\mu(x))' &= \mu(x)(x-3) \\
      \int (u\mu(x))'dx &=\int \mu(x)(x-3)dx \\
      u\mu(x) + c &= \int \mu(x)(x-3)dx \\
      u &= \left(\frac1{\mu(x)}\right)\left(\int \mu(x)(x-3)dx - c\right) \\
    \end{align*}
    $4\mu(x) = \mu'(x)$ has a trivial solution $\mu(x) = e^{-4x}$,
    which when substituted to the above yields
    \begin{align*}
      u &= \left(\frac1{e^{-4x}}\right)\left(\int e^{-4x}(x-3)dx - c\right) \\
      u &= \left(e^{4x}\right)\left(\int e^{-4x}(x-3)dx - c\right) \\
    \end{align*}
    The integral on the right hand side can be solved by integration
    by parts, picking $u = x-3$ and $dv = e^{-4x} \implies v =
    -\frac{1}4e^{-4x}$.
    \begin{align*}
      u &= \left(e^{4x}\right)\left(\int e^{-4x}(x-3)dx - c\right) \\
      u &= e^{4x}\left(\frac{-(x-3)e^{-4x}}{4} - \int \frac{e^{-4x}}{4}
      dx - c\right) \\
      u &= -\frac14e^{4x}\left((x-3)e^{-4x} - \int e^{-4x} dx + c\right) \\
      u &= -\frac14e^{4x}\left((x-3)e^{-4x} +  \frac14e^{-4x} dx + c\right) \\
      u &= -\frac14(x-3) - \frac1{16} + e^{4x}c \\
      u &= -\frac{x}4 + \frac{11}{16} + e^{4x}c
    \end{align*}
  \item $$5u''  - 4u' + 4u = e^x \cos x$$
    Can be rewritten as
    \[
      u'' - \frac45u' + \frac45u = \frac15e^x \cos x
    \]
    First we solve for the complementary solution:
    \begin{align*}
      u'' - \frac45u'  + \frac45u &= 0 \implies \\
      r^2 - \frac45r  + \frac45 &= 0 \\
    \end{align*}
    Which has roots $\frac4{10} - \frac8{10}i$ and $\frac4{10} +
    \frac8{10}i$. Thus the complimentary solution is
    \begin{align*}
      y_c(x) &= c_1e^{(\frac4{10} - \frac8{10}i)x}  + c_2 e^{(\frac4{10}
      + \frac{8}{10}i)x} \\
      &= c_1(e^{\frac4{10}x}e^{-\frac8{10}xi}) +
      c_2(e^{\frac4{10}x}e^{\frac8{10}xi}) \\
      &= e^{\frac4{10}x}\left[(c_1 + c_2)\cos(\frac8{10}x) + i(c_2 -
      c_1)\sin(\frac8{10}x)\right] \\
      &= e^{\frac4{10}x}\left[C_1\cos(\frac8{10}x) +
      C_2\sin(\frac8{10}x)\right] \\
    \end{align*}
    Solving for the particular solution:
\end{enumerate}

\problem solve $u_t = x$

Fixing $x$ and integrating with respect to $t$, we get that
\begin{align*}
  \int u_t dt &= \int x dt \\
  u + c(x) &= xt \\
  u &= xt + c(x)
\end{align*}
For any function of $x$, $c(x)$
\newpage

\end{document}
