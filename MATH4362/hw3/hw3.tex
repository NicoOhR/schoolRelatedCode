% This is a template for doing homework assignments in LaTeX

\documentclass{article} % This command is used to set the type of document you are working on such as an article, book, or presenation

\usepackage{geometry} % This package allows the editing of the page layout
\usepackage{amsmath}  % This package allows the use of a large range of mathematical formula, commands, and symbols
\usepackage{graphicx}  % This package allows the importing of images

\usepackage{
  amsmath,      % Math Environments
  amssymb,      % Extended Symbols
  enumerate,        % Enumerate Environments
  graphicx,      % Include Images
  lastpage,      % Reference Lastpage
  multicol,      % Use Multi-columns
  multirow      % Use Multi-rows
}
\newcommand{\question}[2][]{\begin{flushleft}
        \textbf{Question #1}: \textit{#2}

\end{flushleft}}
\newcommand{\sol}{\textbf{Solution}:} %Use if you want a boldface solution line
\newcommand{\maketitletwo}[2][]{\begin{center}
        \Large{\textbf{Assignment #1}
            
            Course Title} % Name of course here
        \vspace{5pt}
        
        \normalsize{Matthew Frenkel  % Your name here
        
        \today}        % Change to due date if preferred
        \vspace{15pt}
        
\end{center}}
\begin{document}
    \maketitletwo[2]  % Optional argument is assignment number
    %Keep a blank space between maketitletwo and \question[1]
    
    \question[1]{ } 
    Solve 
    \[
      u_{tt} = u_{xx}
    \]
    Subject to 
    $$
        u(0,x) = 
    \begin{cases}
        1, \quad 1 < x < 2 \\ 
        0 \quad \text{otherwise}
    \end{cases}
    $$
    and 

    \[
        u_t(0, x)  = 0 
    \]

    The general solution to the wave equation is given by solving theorem 2.14 
    \[
    u(t,x) = p(x - t) + q(x+t)
    \]
    Subject to the above contraint, we have that 
    \begin{align}
        u(0, x)& = p(x) + q(x) = 1 \quad 1 < x < 2 \\
        u_t(t, x) &= (-1)p'(x - t) + q'(x +t) = 0 \\
        u_t(0,x) &= q'(x) - p'(x) = 0 \implies q'(x) = p'(x)
    \end{align}
    Which means that 
    \[
        \int q'(x) dx = \int p'(x) dx \implies q(x) = p(x) + C \implies q(x) - p(x) = C_p
    \]

    Thus, combining with condition $1$, we have that 
    \[
    2q(x) = 1 + C_p \implies
    q(x) = \frac{1+C_p}{2} 
    \]
    Solving for $p$
    $$
    \frac{1+C}{2} + q(x) = 1
    $$

    \question[2]{Here is my second question}
    
    YOUR SOLUTION HERE
    
    \question[3]{What is the \Large{$\int_0^2 x^2 \, dx $}\normalsize{. Show all steps}}
    
    \begin{align*}
    \int_0^2 x^2 &= \left. \frac{x^3}{3} \right|_0^2 \\
                 &= \frac{2^3}{3}-\frac{0^3}{3}\\
                 &= \frac{8}{3}
    \end{align*}
\end{document}
