% This is a template for doing homework assignments in LaTeX

\documentclass{article} % This command is used to set the type of
% document you are working on such as an article, book, or presenation

\usepackage{geometry} % This package allows the editing of the page layout
\usepackage{amsmath}  % This package allows the use of a large range
% of mathematical formula, commands, and symbols
\usepackage{graphicx}  % This package allows the importing of images
\usepackage{float}  % This package allows precise figure placement
\usepackage{subcaption}  % This package allows subfigure captions

\usepackage{
  amsmath,      % Math Environments
  amssymb,      % Extended Symbols
  enumerate,        % Enumerate Environments
  graphicx,      % Include Images
  lastpage,      % Reference Lastpage
  multicol,      % Use Multi-columns
  multirow      % Use Multi-rows
}
\newcommand{\question}[2][]{
  \begin{flushleft}
    \textbf{Question #1}: \textit{#2}

\end{flushleft}}
\newcommand{\sol}{\textbf{Solution}:} %Use if you want a boldface solution line
\newcommand{\maketitletwo}[2][]{
  \begin{center}
    \Large{\textbf{Assignment #1}

    PDE} % Name of course here
    \vspace{5pt}

    \normalsize{N. Ohayon Rozanes  % Your name here

    \today}        % Change to due date if preferred
    \vspace{15pt}

\end{center}}
\begin{document}
\maketitletwo[3]  % Optional argument is assignment number
%Keep a blank space between maketitletwo and \question[1]

\question[1]{ }
Solve
\[
  u_{tt} = u_{xx}
\]
Subject to
$$
u(0,x) = f(x) =
\begin{cases}
  1, \quad 1 < x < 2 \\
  0 \quad \text{otherwise}
\end{cases}
$$
and

\[
  u_t(0, x) = g(x) =  0
\]

The general solution to the wave equation is given by solving theorem 2.15
\[
  u(t,x) = \frac{f(x-ct) + f(x+ct)}{2} + \frac1{2c}\int_{x-ct}^{x+ct} g(x)
  = \frac{f(x-t) + f(x+t)}{2}
\]
Subject to the above contraint, we have that
\begin{align*}
  f(x-t) &=
  \begin{cases}
    1, \quad 1 + t < x  < 2 + t\\
    0 \quad \text{otherwise}
  \end{cases}\\
  f(x+t) &=
  \begin{cases}
    1, \quad 1 - t< x < 2 - t \\
    0 \quad \text{otherwise}
  \end{cases} \\
  f(x-t) + f(x+t) &=
  \begin{cases}
    1, \quad 1 + t < x < 2 + t \\
    0, \quad \text{otherwise}
  \end{cases}
  +
  \begin{cases}
    1, \quad 1 - t< x < 2 - t \\
    0 \quad \text{otherwise}
  \end{cases} \\
  \frac{f(x-t) + f(x+t)}{2} &=
  \begin{cases}
    \frac{1}{2}, \quad 1 + t < x < 2 + t \\
    0, \quad \text{otherwise}
  \end{cases}
  +
  \begin{cases}
    \frac{1}{2}, \quad 1 - t< x < 2 - t \\
    0 \quad \text{otherwise}
  \end{cases} \\
\end{align*}

Therefore, the solution to this PDE is a square at time $0$ with
height $1$, as expected from the initial condition, which splits into
two squares traveling in opposite directions as t increases, an
illustration of the wave at $t=0, 0.25, 0.5,0.75$, is shown below:

\begin{figure}[H]
  \centering
  \begin{subfigure}[b]{0.45\textwidth}
    \centering
    \includegraphics[width=\textwidth]{t=0.png}
    \caption{$t = 0$}
  \end{subfigure}
  \begin{subfigure}[b]{0.45\textwidth}
    \centering
    \includegraphics[width=\textwidth]{t25.png}
    \caption{$t = 0.25$}
  \end{subfigure}

  \begin{subfigure}[b]{0.45\textwidth}
    \centering
    \includegraphics[width=\textwidth]{t5.png}
    \caption{$t = 0.5$}
  \end{subfigure}
  \begin{subfigure}[b]{0.45\textwidth}
    \centering
    \includegraphics[width=\textwidth]{t75.png}
    \caption{$t = 0.75$}
  \end{subfigure}
\end{figure}

\question[2]{ }

Solve
$$
u_{tt} - u_{xx} - \frac{2}{x}u_x = 0
$$

Subject to:
\[
  u(0, x) = 0 \quad u_t(0,x)= g(x) \quad g(x) = g(-x)
\]

First, set $w(t, x) = xu(t,x)$, which as derivatives:
\[
  w_t = xu_t \quad w_x = xu_x + u \quad w_{tt} = xu_{tt} \quad w_{xx}
  = xu_{xx} + 2u_x
\]

Adding select terms:
$$
w_{tt} - w_{xx} = x(u_{tt} - u_{xx} - \frac{2}{x}u_x)
$$

Thus we can solve the wave equation:

\[
  w_{tt} - w_{xx} = 0
\]

Subject to $w_t = xg(x)$ and $w(0, x) = xu(0,x) = 0$, because the
initial position of the wave is $0$, the first term of d'Almberts
solution disappears, leaving only the integral:
\[
  w(t,x) = \frac{1}{2}\int_{x-t}^{x+t} zg(z) dz = \int_0^{x+t} zg(z)dz
\]

Then, returning to $u$, we get the final solution:
\[
  u(t,x) = \frac{1}{x}w(t,x) =
  \frac{1}{x}\int_0^{x+t} zg(z)dz
\]
\pagebreak
\question[3]{ }
\begin{enumerate}
  \item Find fourier series of $f(x) = |x|$, the function is even,
    and therefore the $b_n$ coefficients vanish.
    \begin{align*}
      a_0 &= \frac{1}{2\pi}\int_{-\pi}^\pi |x|dx \\
      &= \frac{1}{2\pi}\left(\int_{-\pi}^0 -x dx + \int_0^\pi x dx\right)\\
      &= \frac{1}{2\pi}(\frac{\pi^2}{2} + \frac{\pi^2}{2}) \\
      &= \frac{\pi}{2}
    \end{align*}
    Since $|x|$ and $\cos$ are even functions, their product is even
    as well, so we can change the bounds of the integration to be a
    little simpler:
    \begin{align*}
      a_n &= \frac{1}{\pi}\left(\int_{-\pi}^{\pi} |x|\cos(nx)dx\right) \\
      &= \frac{2}{\pi}\left(
      \int_0^\pi x\cos(nx) dx\right) \\
      &= \frac{2}{\pi}\left(
      \int_0^\pi x\cos(nx) dx\right)
    \end{align*}
    Evaluating the integral of $x\cos(nx)$ seperately by integration
    by parts, let $u = x, du = dx, dv = cos(nx), v = \frac{1}{n}sin(nx)$
    \begin{align*}
      \int x\cos(nx)dx &= \frac{x\sin(nx)}{n} - \frac{1}{n}\int \sin(nx)dx \\
      &= \frac{x\sin(nx)}{n} + \frac{1}{n^2}\cos(nx) \\
    \end{align*}
    Then evaluating in the original:
    \[
      a_n = \frac{1}{\pi}(\frac{1}{n^2}(\cos(n\pi)-1)) = \frac{(-1)^n
      - 1}{\pi{n^2}}
    \]
    Thus the fourier series becomes:

    \[
      |x|\sim \frac{\pi}{2}  + \sum_{n\text{ odd}} \frac{-4}{\pi{n^2}}\cos(nx)
    \]

  \item $f(x) = 3x - 1$
    \begin{align*}
      a_0 &= \frac{1}{2\pi} \int_{-\pi}{\pi}3x-1dx \\
      &= \frac{1}{2\pi}\left(\frac{3}{2}x^2 - x\right)\Big|_{-\pi}^\pi\\
      &= \frac{1}{2\pi}(-2\pi) \\
      &= -1
    \end{align*}
    \begin{align*}
      a_n &= \frac{1}{\pi}\int_{-\pi}^\pi \cos(nx)(3x-1)dx \\
      &= \frac{1}{\pi}\left(\int 3x\cos(nx)dx - \int_{-\pi}^\pi
      \cos(nx)dx\right)
    \end{align*}
    Since $3x$ is odd and $\cos(nx)$ is even, their product is odd,
    and therefore the integral over a symmetric domain of
    $3x\cos(nx)$ is equal to $0$. Thus
    \[
      a_n = \frac{-1}{n\pi}\sin(nx)\Big|_{-\pi}^\pi = 0
    \]

    let $u = 3x - 1, du = 3, dv = \sin(nx), v = \frac{-1}{n}\cos(nx)$
    \begin{align*}
      b_n &= \frac{1}{\pi} \int_{-\pi}^{\pi} \sin(nx)(3x-1) dx\\
      &= \frac{1}{\pi}\left(\frac{-(3x-1)\cos(nx)}{n} +
      \frac{3}{n}\int \cos(nx) dx\right) \\
      &= \frac{1}{\pi}\left(\frac{-(3x-1)}{n}\cos(nx) +
      \frac{3}{n^2}\sin(nx) dx\right)\Big|_{-\pi}^{\pi} \\
      &= \frac{1}{\pi}\left(\frac{-(3\pi-1)}{n} -
      \frac{-(-3\pi-1)}{n}\right)(-1)^n \\
      &= \frac{6(-1)^{n+1}}{n} \\
    \end{align*}
    Thus
    \[
      3x - 1 \sim -1 + \sum_n \frac{6}{n}(-1)^{n+1}\sin(nx)
    \]
\end{enumerate}
\end{document}
