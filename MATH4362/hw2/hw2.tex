\documentclass{article} % This command is used to set the type of
% document you are working on such as an article, book, or presenation
\usepackage{
  amsmath,      % Math Environments
  amssymb,      % Extended Symbols
  enumerate,        % Enumerate Environments
  graphicx,      % Include Images
  lastpage,      % Reference Lastpage
  multicol,      % Use Multi-columns
  multirow      % Use Multi-rows
}

\usepackage{geometry} % This package allows the editing of the page layout
\usepackage{amsmath}  % This package allows the use of a large range
% of mathematical formula, commands, and symbols
\usepackage{graphicx}  % This package allows the importing of images

\newcommand{\question}[2][]{
  \begin{flushleft}
    \textbf{Question #1}: \textit{#2}

\end{flushleft}}
\newcommand{\sol}{\textbf{Solution}:} %Use if you want a boldface solution line
\newcommand{\maketitletwo}[2][]{
  \begin{center}
    \Large{\textbf{Assignment #1}

    Course Title} % Name of course here
    \vspace{5pt}

    \normalsize{Matthew Frenkel  % Your name here

    \today}        % Change to due date if preferred
    \vspace{15pt}

\end{center}}
\begin{document}
\maketitletwo[2]  % Optional argument is assignment number
%Keep a blank space between maketitletwo and \question[1]

\question[2.2.20]{ }

\[
  u_t + (1+x^2)u_x = 0
\]
\begin{enumerate}[(a)]
  \item In this problem, $c(t) = 1 + t^2$, thus the characteristic
    curve solves the ODE:
    $$
    \frac{d\mathbf{x}}{dt} = c(\mathbf{x}(t)) = 1 + \mathbf{x}(t)^2
    $$
    Which is a first order seperable differential equation:
    \[
      \frac{d\mathbf{x}}{dt}\frac{1}{1 + \mathbf{x}(t)^2} = 1
    \]
    let $y = x(t)$ thus $dy = \frac{d\mathbf{x}}{dt} dx$
    \begin{align*}
      \frac{1}{1 + y(t)^2} dy&= 1 \\
      \arctan(y(t)) &= t + C \\
      \arctan(y(t)) - t &= C \\
    \end{align*}
    Thus the characterisitic variable is $$\beta(t) = \arctan(t) - t$$
    Fully solving the above, get that the characteristic curve $$x(t)
    = \tan(C +t)$$ for some constant $C$
  \item  Thus the solutions to the PDE are any function $v$ such that
    $u(t,x) = v(\arctan(x) - t)$
  \item For initial condition $u(0, x) = f(x)$, consider that the
    solution at $t = 0$ would be a function of only $\arctan(x)$ that is
    \[
      u(0, x) = v(\arctan(x)) = f(x)
    \]
    Notice that $v(x) := \tan(x)$ gives $v(\arctan(x)) =
    x$, therefore:
    \[
      v(x) := f(\tan(x)) \implies u(t, x) = v(\arctan(x) - t) =
      f(\tan(\arctan(x) - t))
    \]
    will solve the initial value problem. As $t$ increases,
    $\tan(\arctan(x) - t)$ will change much more with $t$ than with
    $x$, which is to say that the solution to the PDE is more
    sensitive to chnanges in time than in position.
\end{enumerate}

\question[2.2.23]{ }

YOUR SOLUTION HERE

\question[2.2.26]{ }

YOUR SOLUTION HERE

\end{document}
