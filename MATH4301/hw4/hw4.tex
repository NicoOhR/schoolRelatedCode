\documentclass[12pt]{article}

\usepackage[utf8]{inputenc}
\usepackage{latexsym,enumi, amsfonts,amssymb,amsthm,amsmath}

\setlength{\parindent}{0in}
\setlength{\oddsidemargin}{0in}
\setlength{\textwidth}{6.5in}
\setlength{\textheight}{8.8in}
\setlength{\topmargin}{0in}
\setlength{\headheight}{18pt}



\title{Math 333 Weekly Homework X}
\author{Your Name Here}

\begin{document}

\maketitle

\vspace{0.5in}



\subsection*{Exercise 1}
\begin{enumerate}[label = \alph*)]
  \item $\Sup A \in \mathbb{R} \leftrightarrow A$ is bounded, and $A \ne \varnothing$. Since the sequence converges, we can say that the sequence is Cauchy. Since every Cauchy sequence is bounded, and $A$ obviously is not empty, it must have $\sup A \in \mathbb{R}$ 
  \item \begin{proof}
    For increasing converging sequence $a_n$, the following conditions hold by definition:
    \begin{align}
      a_{n+1} \ge a_n \quad& \forall n \in \mathbb{N}\\ 
      \exists N \in \mathbb{N}: \lvert a - a_n \rvert \le \epsilon \quad \forall n \ge N \epsilon > 0
    \end{align}
     By definition, $\forall \delta > 0$ then $\exists a_N \in A : \sup A - \delta < a_N \implies \sup A - a_n < \delta$. Since the element $a_N \in A$ is the $N$th element of the sequence $a_n$, and $a_n$ is increasing, we can say $\sup A - a_m < \sup A - a_N < \delta \forall m > N$. This is exactly the definition of $\lim a_n = \sup A$
  \end{proof}
\end{enumerate}

\subsection*{Exercise 2}
The sequence can be defined recursively as 
\[
  a_1 = \sqrt{2} \quad a_{n+1} = \sqrt{2\a_n}
\]
\begin{proof}
(Type your proof here.)
\end{proof}

\vspace{2in} %Leave more space for comments!







\end{document}

