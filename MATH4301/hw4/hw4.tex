\documentclass[12pt]{article}


\usepackage[utf8]{inputenc}
\usepackage{latexsym,enumitem, amsfonts,amssymb,amsthm,amsmath}

\setlength{\parindent}{0in}
\setlength{\oddsidemargin}{0in}
\setlength{\textwidth}{6.5in}
\setlength{\textheight}{8.8in}
\setlength{\topmargin}{0in}
\setlength{\headheight}{18pt}
\renewcommand{\epsilon}{\varepsilon}


\title{N. Ohayon Rozanes}
\author{N. Ohayon Rozanes}

\begin{document}

\maketitle

\vspace{0.5in}



\subsection*{Exercise 1}
\begin{enumerate}[label = \alph*)]
  \item $\sup A \in \mathbb{R} \rightleftarrows A$ is bounded, and $A \ne \emptyset$. Since the sequence converges, we can say that the sequence is Cauchy. Since every Cauchy sequence is bounded, and $A$ obviously is not empty, it must have $\sup A \in \mathbb{R}$ 
  \item \begin{proof}
    For increasing converging sequence $a_n$, the following conditions hold by definition:
    \begin{align}
      a_{n+1} \ge a_n \quad& \forall n \in \mathbb{N}\\ 
      \exists N \in \mathbb{N}: \lvert a - a_n \rvert \le \epsilon \quad \forall n \ge N \epsilon > 0
    \end{align}
     By definition, $\forall \delta > 0$ then $\exists a_N \in A : \sup A - \delta < a_N \implies \sup A - a_n < \delta$. Since the element $a_N \in A$ is the $N$th element of the sequence $a_n$, and $a_n$ is increasing, we can say $\sup A - a_m < \sup A - a_N < \delta \forall m > N$. This is exactly the definition of $\lim a_n = \sup A$
  \end{proof}
\end{enumerate}

\subsection*{Exercise 2}
The sequence can be defined recursively as 
$$
  a_1 = \sqrt{2} \quad a_{n+1} = \sqrt{2a_n}
$$
\begin{proof}
  Boundedness is demonstrated with induction, let the upper bound be $2$
  \begin{itemize}
    \item Base case, $a_1 < 2$ 
    \item assume that $a_n < 2$ 
    \item Since $a_{n+1} = \sqrt{2a_n} \implies \frac12a_{n+1}^2 = a_n$, then we can say $\frac12 a^2_{n+1} < 2 \implies a_{n+1} < \sqrt{2 \cdot 2} \implies a_{n+1} < 2 $
  \end{itemize}
  To show that the sequence is increasing, we show that the inequality $a_{n+1} \le a_n$ holds 
  \begin{align*}
    a_{n+1} &\ge a_n \\
    \sqrt{2a_n}&\ge a_n \\
    2a_n &\ge a_n^2 \\ 
  \end{align*}
  The above holds for all $a_n \le 2$, and we have shown $a_n < 2 \forall n \in \mathbb{N}$. Because the sequence is both increasing and bounded above, by MCT it is convergent. Let the limit of this sequence be $a_n \to a$. Then, $$\lim a_{n+1} = \lim \sqrt{2a_n} \implies a = \sqrt{2a}$$ in another way $$ a^2 - 2a = 0$$ which has only one real solution $a = 2$
\end{proof}
\subsection*{Exercise 3}
\begin{proof}
  Construct $I_n = [a_n, b_n]$, such that \[
    I_n \subseteq I_{n+1} 
  \]
  The sequence of lower bounds $a_n$ of the interval sequence must be increasing and the sequence of lower bounds must be decreasing since $$[a_n, b_n] \subseteq [a_{n+1}, b_{n+1}] \implies a_n \le a_{n+1}, b_n \ge b_{n+1}$$ 
  The sequence $a_n$ is also by construction bounded above by any element of $b_n$ and $b_n$ is bounded below by any element of $a_n$. By the MCT, $\lim a_n = \sup \{a_n \mid  \forall n \in \mathbb{N}\}$. Therefore, $$a_n \le s \le b_n \quad \forall n \in \mathbb{N}$$ So: 
  \[
    s \in \bigcap_{n=0}^\infty I_n 
  \]
\end{proof}

\subsection*{Exercise 4}
\begin{enumerate}[label=\alph*)]
  \item Since $a_n$ is bounded, that means that $\exists M \in \mathbb{R}: \forall n\in\mathbb{N}$, $a_n < M$. Then, $\forall x \in S \implies x < a_n < M$ so $M$ is an upper bound of the set as well, thus the set is bounded.
\item For $s := \sup S$ then $\exists x \in S :  s - \epsilon < x$ for any $\epsilon > 0$. Also, since $x < a_n$ for an infinite number of $n \in \mathbb{N}$ then $s - \epsilon < a_n \implies s - a_n < \epsilon$ for all such $n$. Because there is an infinite number of them, if we simply subsequence $n_k$ to be strictly the subsequence for which the above inequality holds, then $\lvert s - a_{n-k} < \epsilon$ which is exactly the definition of $a_{n-k} \to s$ 
  \end{enumerate}
\subsection*{Exercise 5}
\begin{proof} 
  Fix $m > n$
\[
  \lvert x_n - x_m \rvert  = \lvert x_n - x_{n+1} + x_{n+1} - x_m \rvert
\]
  Continue subtracting and then adding successive terms of $x_n$ until $x_m-1$ is reached
  \[
    \lvert x_n - x_{n+1} + x_{n+1} - x_m \rvert = \lvert x_n  - x_{n+1} + x_{n+1} - x_{n+2} + x_{n+2} ... + x_{m-1} - x_{m} \rvert
  \]
  Then by the triangle inequality
  \begin{align*}
    \lvert x_n  - x_{n+1} + x_{n+1} - x_{n+2} + x_{n+2} ... + x_{m-1} - x_{m}\rvert &\le \lvert x_n - x_{n+1} \rvert + ... \lvert x_{m-1} - x_{m} \rvert \\
                                                                                    &\le \frac1{3^n} + \frac1{3^{n+1}} + ... \frac{1}{3^m}\\
  \end{align*}
  This is a geometric series of the form 
  \[
    \sum_{k=n}^m (\frac13)^k
  \]
  which evaluates to 
  \begin{align*}
    \frac1{3^n}\left(\frac{1-\frac13^{m-n}}{{1-\frac13}}\right) \\
    \frac32\cdot\frac1{3^n}(1 - \frac1{3^{m-n}}) \\
  \end{align*}
  This is strictly less than $\frac32\frac1{3^n}$ since $\lvert1 - \frac1{3^{m-n}} \rvert< 1$. Solving for the minimum $n$ such that 
  \begin{align*}
    \frac32\frac1{3^n} &< \epsilon\\
    \frac3{2\epsilon} &< 3^n \\
    \log_3{3} - \log_3{2} + \log_3{\frac1\epsilon} &< n
  \end{align*}
  So for any $m, n > \lceil \log_3{3} - \log_3{2} + \log_3{\frac1\epsilon} \rceil$ \[
    \lvert x_n - x_m \rvert < \epsilon
  \]
  Therefore the sequence is Cauchy and by the Cauchy criterion convergent as well.
\end{proof}
\end{document}

