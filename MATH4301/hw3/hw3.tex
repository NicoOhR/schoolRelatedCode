\documentclass[12pt]{article}

\usepackage[utf8]{inputenc}
\usepackage{latexsym,amsfonts,amssymb,amsthm,amsmath}

\setlength{\parindent}{0in}
\setlength{\oddsidemargin}{0in}
\setlength{\textwidth}{6.5in}
\setlength{\textheight}{8.8in}
\setlength{\topmargin}{0in}
\setlength{\headheight}{18pt}
\let\epsilon\varepsilon


\title{Homework 3}
\author{N. Ohayon Rozanes}

\begin{document}

\maketitle

\vspace{0.5in}



\subsection*{Exercise 1}
\begin{enumerate}
  \item 
  For $| | x | - | y | | \leq |x - y|$, then 
  $-\lvert x - y\rvert \leq \lvert x \rvert - \lvert y \rvert \leq \lvert x - y \rvert$
\begin{proof}
\begin{align*}
  \lvert x \rvert &= \lvert x - y + y \rvert  \\
  \lvert x \rvert &\leq \lvert x - y \rvert + \lvert y \rvert \\
  \lvert x \rvert - \lvert y \rvert &\leq \lvert x - y \rvert 
\end{align*}
And
\begin{align*}
  \lvert y \rvert &= \lvert y - x + x \rvert  \\
  \lvert y \rvert &\leq \lvert y - x \rvert + \lvert x \rvert \\
  \lvert y \rvert - \lvert x \rvert &\leq \lvert y - x \rvert \\
  -1(\lvert x \rvert - \lvert y \rvert) &\leq \lvert y - x \rvert \\
  \lvert x \rvert - \lvert y \rvert &\geq -\lvert y - x \rvert
\end{align*}
And since:
\begin{align*}
  \lvert x - y \rvert &= \lvert -1(y - x) \rvert  \\
                       &= \lvert y - x \rvert 
\end{align*}
We combine $\lvert x\rvert - \lvert y \rvert \leq \lvert x - y \rvert$ 
and $\lvert x \rvert - \lvert y \rvert \geq - \lvert y - x \rvert$, 
resulting in 
\[
-\lvert x - y\rvert \leq \lvert x \rvert - \lvert y \rvert \leq \lvert x - y \rvert,
\]
which implies
\[
| | x | - | y | | \leq | x - y |
\]
as desired.
\end{proof}
\item If $\lim_{n \to \infty} a_n = a$ then $\forall \epsilon, \exists N \in \mathbb{N} : \forall n \geq N, \lvert a_n - a \rvert \leq \epsilon$. Using the reverse triangle inequality, $\epsilon \geq \lvert  a_n - a \rvert \geq \lvert \lvert a_n \rvert - \lvert a \rvert \rvert$. Therefore, for any positive $\epsilon$, the same lower bound $N$ will satisfy both $\lvert a_n - a \rvert \leq  \epsilon$ and $\lvert\lvert a_n \rvert - \lvert a \rvert\rvert \leq \epsilon$ which implies that $\lim\limits_{n\to\infty} \lvert a_n\rvert = \lvert a \rvert$.  
\item False, consider 
  \[
    \{ a_n \}^\infty_{n=1} =
    \begin{cases}
      1, &\text{ if } n \text{ is even}\\
      -1, &\text{ if } n \text{ is odd}\\
    \end{cases}
  \]
  where $\lvert a_n\rvert$ will converge to $1$, for any $N$. But for any $\epsilon < 2$ no such $N$ can be found for $a_n$.
\end{enumerate}
\subsection*{Exercise 2}
Want to show: if $\lim a_n = a$ then, $\forall \delta > 0, \exists N \in \mathbb{N} : \forall n > N, \lvert a_n^3  - a^3 \rvert < \delta$ 
\begin{proof}
  \begin{align*}
a_n^3 - a^3
&= a_n^3 - a_n^2a + a_n^2a - a_na^2 + a_na^2 - a^3 \\
&= (a_n^3 - a_n^2a) + (a_n^2a - a_na^2) + (a_na^2 - a^3) \\
&= a_n^2(a_n - a) + a_na(a_n - a) + a^2(a_n - a) \\
&= (a_n - a)(a_n^2 + a_na + a^2).
\end{align*}
So \[
  \lvert a^3_n - a^3 \rvert = \lvert a_n - a\rvert\lvert a_n^2 + a_na + a^2\rvert
\]
Because $\lvert a_n - a \rvert < \epsilon$ then: 
\[
  \lvert a^3_n - a^3 \rvert = \lvert a_n - a\rvert\lvert a_n^2 + a_na + a^2\rvert < \epsilon\lvert a_n^2 + a_na + a^2\rvert
\]
Further, since $a_n \to a$, we can find $N_1$ such that $\lvert a_n - a\rvert < 1$, so 
\begin{align*}
  \lvert a_n \rvert &= \lvert a_n - a + a \rvert \\ 
                    &\le \lvert a_n - a \rvert + \lvert a\rvert \\
                    &\le 1 + \lvert a \rvert
\end{align*}
Which allows us to introduce an upper bound not dependent on $n$ on $\lvert a_n^2 + a_na + a^2\rvert$
\begin{align*}
  \lvert a_n^2 + a_na + a^2 \rvert  &\leq \lvert a_n^2 \rvert + \lvert a_n a\rvert + \lvert a^2 \rvert \\
                                    &\le \lvert (\lvert a \rvert + 1)^2 \rvert + \lvert \lvert a + 1 \rvert \rvert a \rvert \rvert+ \lvert a^2\rvert \\
                                    &= \lvert (\lvert a \rvert^2 + 2\lvert a \rvert+ 1) \rvert + \lvert \lvert a \rvert^2  + \rvert a \rvert\rvert+ \lvert a^2\rvert \\
                                    &\le 3\lvert a \rvert^2 + 3\lvert a \rvert + 1
\end{align*}
So \[
  \lvert a_n^3 - a^3 \rvert \le \epsilon(3\lvert a\rvert^2 + 3\lvert a\rvert + 1)
\]
Set $C:= 3\lvert a\rvert^2 + 3\lvert a\rvert + 1$. Now set $\varepsilon:=\delta/C$ (note $C\ge 1$, so $\varepsilon>0$). Since $a_n\to a$, choose $N_1$ such that for all $n\ge N_1$, $|a_n-a|<\varepsilon$.
For $n\ge N:=\max\{N_0,N_1\}$ we have
\[
|a_n^3-a^3|
=|a_n-a|\cdot|a_n^2+a_na+a^2|
\le \varepsilon\, C
=\delta.
\]
Therefore $a_n^3\to a^3$.
\end{proof}
\subsection*{Exercise 3}
\begin{proof}
  \begin{align*}
    \lvert\sqrt{c + x_n} - \sqrt{c + x}\rvert &= \frac{\lvert\sqrt{c + x_n} + \sqrt{c + x}\rvert}{\lvert\sqrt{c + x_n} + \sqrt{c + x}\rvert}\lvert\sqrt{c + x_n} - \sqrt{c + x}\rvert  \\ 
                                              &= \frac{\lvert c + x_n - c - x\rvert}{\lvert\sqrt{c + x_n} + \sqrt{c + x}\rvert}\\
                                              &= \frac{\lvert x_n - x \rvert}{\lvert\sqrt{c + x_n} + \sqrt{c + x}\rvert} \\
                                              &\le \frac{\epsilon}{\lvert\sqrt{c + x_n} + \sqrt{c + x}\rvert}
  \end{align*}
  Since $c + x_n > 0$, $$\lvert\sqrt{c + x_n} + \sqrt{c + x}\rvert \geq \lvert \sqrt{c + x}\rvert \implies \frac{1}{\lvert\sqrt{c + x_n} + \sqrt{c + x}\rvert} \leq \frac{1}{\lvert \sqrt{c + x}\rvert}$$
  So 
  \[
    \lvert\sqrt{c + x_n} - \sqrt{c + x}\rvert  \le \frac{\epsilon}{\sqrt{c + x}}
  \]
Since $x_n\to x$ and $c+x>0$, there exists $N_0$ such that for all $n\ge N_0$,
\[
|x_n-x|<\tfrac{c+x}{4}\qquad\Rightarrow\qquad c+x_n\ge c+x - |x_n-x|>\tfrac{c+x}{2}>0,
\]
hence
\[
\sqrt{c+x_n}+\sqrt{c+x}\;\ge\;\sqrt{\tfrac{c+x}{2}}+\sqrt{c+x}\;\ge\;\tfrac12\sqrt{c+x}.
\]
Therefore, for $n\ge N_0$,
\[
\bigl|\sqrt{c+x_n}-\sqrt{c+x}\bigr|
=\frac{|x_n-x|}{\sqrt{c+x_n}+\sqrt{c+x}}
\le \frac{2}{\sqrt{c+x}}\,|x_n-x|.
\]
Now let $\delta>0$ be given. Since $x_n\to x$, there exists $N_1$ such that for all $n\ge N_1$,
\[
|x_n-x|<\frac{\delta\,\sqrt{c+x}}{2}.
\]
Taking $N=\max\{N_0,N_1\}$, for all $n\ge N$ we have
\[
\bigl|\sqrt{c+x_n}-\sqrt{c+x}\bigr|
\le \frac{2}{\sqrt{c+x}}\,|x_n-x|
< \frac{2}{\sqrt{c+x}}\cdot \frac{\delta\,\sqrt{c+x}}{2}
=\delta.
\]
Hence $\sqrt{c+x_n}\to\sqrt{c+x}$.
\end{proof}
\subsection*{Exercise 4}
\begin{proof}
  After calcuating the first couple of terms, the sequence seems to tend to $2$, working from that hypothesis, set $x = 2$. Want to show that the sequence is convergent by showing that it is increasing, as well as bounded. First we prove that $0 < x_n < 2 \forall n \in \mathbb{N}$ using induction:
  \begin{enumerate}
    \item Base case $0 < x_1 < 2$ 
    \item assume $0 < x_n < 2$ 
    \item Since the square root function is increasing for all positive reals and has a codomain of $[0, \infty)$, then $x_{n+1} = \sqrt{2 + x_n} < \sqrt{2 + 2} \implies x_{n+1} < 2$and $x_{n+1} = \sqrt{2 + x_n} > 0$ because $x_n + 2 > 0$ by assumption. 
  \end{enumerate}
  Then we show that it is increasing
  \begin{align*}
    x_{n+1} &\ge x_n \\
    \sqrt{x_n + 2} &\ge x_n \\
    x_n + 2 &\ge x_n^2 \\
    x_n + 2 &\ge x_n^2 \\
    -x_n^2 + x_n + 2 &\ge 0 \\
    -x_n^2 + x_n + 2 &\ge 0 \\
    (x_n+1)(2 - x_n) &\ge 0
  \end{align*}
  And since $0 < x_n < 2 \forall n \in \mathbb{N}$  the inequality holds. Therefore, the sequence is both bounded and increasing, it converges. To find the limit, set $l = \lim x_n$ as $n \to \infty$, then $l = \sqrt{2 + l} \implies l^2 - l - 2 = 0$ Which has solutions $2$ and $-1$, since we know that $x_n > 0$, the limit must be $2$.
\end{proof}
\subsection*{Exercise 5}
False 
\begin{proof}
  We are given that the sequence (call it $x_n$) is increasing and divergent, then by the monotone convergance theorem, we know it must not be bounded above, that is, there does not exist a real number $M$ such that for all $n \in \mathbb{N}$ $x_n < M$ and $x_{n+1} > x_n$.
  \\ \\
  Lets assume for the sake of contradiction that we can construct an increasing integer sequence $n_i$ such that $x_{n_i} \to L$. 
  Notice also that every subsequence of an increasing sequence is also increasing, since $n_{i+1} > n_{i} \implies x_{n_{i+1}} > x_{n_i}$. Since $x_{n_i}$ is convergent and increasing, it must be bounded. That is, $$\exists M_1 \in \mathbb{R} : \forall n_i, x_{n_i} < M_1$$
  However, since both $n$ and $n_i$ are infinite increasing integer sequences, $\forall n, \exists n_i$ such that $n < n_i \implies x_n < x_{n_i} < M_1$. Which implies that $x_n$ is bounded, and therefore is a contradiction, concluding that no such subsequence $n_i$ exists.
\end{proof}
\end{document}

