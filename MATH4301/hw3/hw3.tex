\documentclass[12pt]{article}

\usepackage[utf8]{inputenc}
\usepackage{latexsym,amsfonts,amssymb,amsthm,amsmath}

\setlength{\parindent}{0in}
\setlength{\oddsidemargin}{0in}
\setlength{\textwidth}{6.5in}
\setlength{\textheight}{8.8in}
\setlength{\topmargin}{0in}
\setlength{\headheight}{18pt}
\let\epsilon\varepsilon


\title{Homework 3}
\author{N. Ohayon Rozanes}

\begin{document}

\maketitle

\vspace{0.5in}



\subsection*{Exercise 1}
\begin{enumerate}
  \item 
  For $\lvert \lvert x \rvert - \lvert y \rvert \rvert \leq \lvert x - y \rvert$, then 
  $-\lvert x - y\rvert \leq \lvert x \rvert - \lvert y \rvert \leq \lvert x - y \rvert$
\begin{proof}
\begin{align*}
  \lvert x \rvert &= \lvert x - y + y \rvert  \\
  \lvert x \rvert &\leq \lvert x - y \rvert + \lvert y \rvert \\
  \lvert x \rvert - \lvert y \rvert &\leq \lvert x - y \rvert 
\end{align*}
And
\begin{align*}
  \lvert y \rvert &= \lvert y - x + x \rvert  \\
  \lvert y \rvert &\leq \lvert y - x \rvert + \lvert x \rvert \\
  \lvert y \rvert - \lvert x \rvert &\leq \lvert y - x \rvert \\
  -1(\lvert x \rvert - \lvert y \rvert) &\leq \lvert y - x \rvert \\
  \lvert x \rvert - \lvert y \rvert &\geq -\lvert y - x \rvert
\end{align*}
And since:
\begin{align*}
  \lvert x - y \rvert &= \lvert -1(y - x) \rvert  \\
                       &= \lvert y - x \rvert 
\end{align*}
We combine $\lvert x\rvert - \lvert y \rvert \leq \lvert x - y \rvert$ 
and $\lvert x \rvert - \lvert y \rvert \geq - \lvert y - x \rvert$, 
resulting in 
\[
-\lvert x - y\rvert \leq \lvert x \rvert - \lvert y \rvert \leq \lvert x - y \rvert,
\]
which implies
\[
\lvert \lvert x \rvert - \lvert y \rvert \rvert \leq \lvert x - y \rvert
\]
as desired.
\end{proof}
\item If $\lim_{n \to \infty} a_n = a$ then $\forall \epsilon, \exists N \in \mathbb{N} : \forall n \geq N, \lvert a_n - a \rvert \leq \epsilon$. Using the reverse triangle inequality, $\epsilon \geq \lvert  a_n - a \rvert \geq \lvert \lvert a_n \rvert - \lvert a \rvert \rvert$. Therefore, for any positive $\epsilon$, the same lower bound $N$ will satisfy both $\lvert a_n - a \rvert \leq  \epsilon$ and $\lvert\lvert a_n \rvert - \lvert a \rvert\rvert \leq \epsilon$ which implies that $\lim\limits_{n\to\infty} \lvert a_n\rvert = \lvert a \rvert$.  
\item False, consider 
  \[
    \{ a_n \}^\infty_{n=1} =
    \begin{cases}
      1, &\text{ if } n \text{ is even}\\
      -1, &\text{ if } n \text{ is odd}\\
    \end{cases}
  \]
  where $\lvert a_n\rvert$ will converge to $1$, for any $N$. But for any $\epsilon < 2$ no such $N$ can be found for $a_n$.
\end{enumerate}
\subsection*{Exercise 2}
Want to show: if $\lim a_n = a$ then, $\forall \delta > 0, \exists N \in \mathbb{N} : \forall n > N, \lvert a_n^3  - a^3 \rvert < \delta$ 
\begin{proof}
  \begin{align*}
a_n^3 - a^3
&= a_n^3 - a_n^2a + a_n^2a - a_na^2 + a_na^2 - a^3 \\
&= (a_n^3 - a_n^2a) + (a_n^2a - a_na^2) + (a_na^2 - a^3) \\
&= a_n^2(a_n - a) + a_na(a_n - a) + a^2(a_n - a) \\
&= (a_n - a)(a_n^2 + a_na + a^2).
\end{align*}
So \[
  \lvert a^3_n - a^3 \rvert = \lvert a_n - a\rvert\lvert a_n^2 + a_na + a^2\rvert
\]
Because $\lvert a_n - a \rvert < \epsilon$ then: 
\[
  \lvert a^3_n - a^3 \rvert = \lvert a_n - a\rvert\lvert a_n^2 + a_na + a^2\rvert < \epsilon\lvert a_n^2 + a_na + a^2\rvert
\]
Further, since $a_n \to a$, we can find $N_1$ such that $\lvert a_n - a\rvert < 1$, so 
\begin{align*}
  \lvert a_n \rvert &= \lvert a_n - a + a \rvert \\ 
                    &\le \lvert a_n - a \rvert + \lvert a\rvert \\
                    &\le 1 + \lvert a \rvert
\end{align*}
Which allows us to introduce an upper bound not dependent on $n$ on $\lvert a_n^2 + a_na + a^2\rvert$
\begin{align*}
  \lvert a_n^2 + a_na + a^2 \rvert  &\leq \lvert a_n^2 \rvert + \lvert a_n a\rvert + \lvert a^2 \rvert \\
                                    &\le \lvert (\lvert a \rvert + 1)^2 \rvert + \lvert \lvert a \rvert a \rvert+ \lvert a^2\rvert \\
                                    &= 3\lvert a \rvert^2 + 2\lvert a \rvert + 1
\end{align*}
So \[
  \lvert a_n^3 - a^3 \rvert \le \epsilon(3\lvert a\rvert^2 + 2\lvert a\rvert + 1)
\]
So for any $\delta$, choose an $N$ sufficiently large such that $\epsilon(3\lvert a\rvert^2 + 2\lvert a\rvert + 1) < \delta$
%for the love of god make the above and below better phrased jfc
\end{proof}
\pagebreak
\subsection*{Exercise 3}
\begin{proof}
  \begin{align*}
    \lvert\sqrt{c + x_n} - \sqrt{c + x}\rvert &= \frac{\lvert\sqrt{c + x_n} + \sqrt{c + x}\rvert}{\lvert\sqrt{c + x_n} + \sqrt{c + x}\rvert}\lvert\sqrt{c + x_n} - \sqrt{c + x}\rvert  \\ 
                                              &= \frac{\lvert c + x_n - c - x\rvert}{\lvert\sqrt{c + x_n} + \sqrt{c + x}\rvert}\\
                                              &= \frac{\lvert x_n - x \rvert}{\lvert\sqrt{c + x_n} + \sqrt{c + x}\rvert} \\
                                              &\le \frac{\epsilon}{\lvert\sqrt{c + x_n} + \sqrt{c + x}\rvert}
  \end{align*}
  Since $c + x_n > 0$, $$\lvert\sqrt{c + x_n} + \sqrt{c + x}\rvert \geq \lvert \sqrt{c + x}\rvert \implies \frac{1}{\lvert\sqrt{c + x_n} + \sqrt{c + x}\rvert} \leq \frac{1}{\lvert \sqrt{c + x}\rvert}$$
  So 
  \[
    \lvert\sqrt{c + x_n} - \sqrt{c + x}\rvert  \le \frac{\epsilon}{\sqrt{c + x}}
  \]
  And because $x_n \to x$ for any $\delta$, we can find an $N \in \mathbb{N}$ such that $$\frac{\epsilon}{\sqrt{c + x}} \le \delta$$
\end{proof}

\end{document}

