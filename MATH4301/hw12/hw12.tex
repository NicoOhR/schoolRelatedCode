\documentclass[12pt]{article}

\usepackage[utf8]{inputenc}
\usepackage{latexsym,amsfonts,amssymb,amsthm,amsmath}

\setlength{\parindent}{0in}
\setlength{\oddsidemargin}{0in}
\setlength{\textwidth}{6.5in}
\setlength{\textheight}{8.8in}
\setlength{\topmargin}{0in}
\setlength{\headheight}{18pt}

\title{MATH 4301}
\author{N. Ohayon Rozanes}

\begin{document}

\maketitle

\vspace{0.5in}

\subsection*{Exercise 1}
\begin{proof}
  \begin{enumerate}
    \item
      On the interval $[0, a]$, the function $f(x) = x^2$ is strictly
      increasing, so every subinterval $i$ of the partition $P$, we have that
      \[
        m_i := \inf_{[x_{i-1}, x_i]} f = f(x_{i-1})
      \]
      and
      \[
        M_i := \sup_{[x_{i-1}, x_i]} f = f(x_i)
      \]
      Therefore:
      \begin{align*}
        U(f,P) - L(f,P) &= \sum_{i=1}^n (f(x_i)-f(x_{i-1}))(x_i - x_{i-1}) \\
        &= \sum_{i=1}^n (x_i^2 - x_{i-1}^2)(x_i - x_{i-1}) \\
        &= \sum_{i=1}^n (x_i - x_{i-1})(x_i + x_{i-1})(x_i - x_{i-1}) \\
        &= \sum_{i=1}^n (x_i - x_{i-1})^2(x_i + x_{i-1}) \\
      \end{align*}
      the sum $x_i + x_{i-1}$ is bounded by $2a$, further, since we
      are free to choose the partition, we are free to define a
      $\delta > 0$ such that $\forall i, x_i - x_{i-1} < \delta$.
      \begin{align*}
        U(f,P) - L(f,P) &\leq \sum_{i=1}^n \delta^2 2a \\
        &\leq n\delta^22a
      \end{align*}
      So, for $$U(f, P) - L(f,P) < \epsilon$$ choose a $P$ such that $\forall i$

      $$x_i - x_{i-1} < \sqrt{\frac{\varepsilon}{2an}}$$

    \item by the previous part and proposition 4.6.10, we know that
      $f: [0, a]$ is Darboux integrable. That is the upper and lower
      Darboux integrals coincide, and are equal to the integral. We
      want to show that they coincide to $\frac{a^3}{3}$, so it
      suffices to show that $\sup \mathcal{L}(f) = \frac{a^3}{3}$.
      Set $P_n = \{\frac{ia}n\}$ for $i = \{0, 1, ... n\}$.
      \begin{align*}
        L(f, P_n) &= \sum_{i=0}^{n} (\frac{ai}{n})^2(\frac{ai - ai + a}{n}) \\
        &= \sum_{i=0}^{n} (\frac{ai}{n})^2(\frac{a}{n}) \\
        &= \frac{a^3}{n^3}\sum_{i=0}^{n} i^2 \\
        &= \frac{a^3}{n^3}\frac{n(n+1)(2n+1)}{6} \\
        &= \frac{2a^3}{6} + \frac{3a^3}{6n} + \frac{a^3}{6n^2}
      \end{align*}
      as $n \to \infty$
      \[
        L(f, P_n)  = \frac{a^3}{3} = \int_0^a f
      \]
  \end{enumerate}
\end{proof}
\subsection*{Exercise 2}
\begin{proof}
  Want to show that $\sup\mathcal{L}(f) = \inf\mathcal{U}(f)$. It
  suffices to show that for all $\varepsilon > 0$ there exists a
  partition $P$ such that $$U(f, P) - L(f, P)< \varepsilon$$
  For a strictly increasing function $f$, the infimum and supremum of
  the function over the interval $[x_{i-1}, x_i]$ is $f(x_{i-1})$ and
  $f(x_{i})$ respectively. So for any partition of $[a,b]$, $P$,
  \[
    L(f, P) =  \sum  f(x_{i-1})(x_i - x_{i-1})
  \]
  And
  \[
    U(f, P) =  \sum  f(x_{i})(x_i - x_{i-1})
  \]
  so
  \[
    U(f, P) - L(f, P) =  \sum  (f(x_{i}) - f(x_{i-1})(x_i - x_{i-1})
    \]
    Since $x_i > x_{i-1}$, and that $f$ is an increasing function, we
    have that $f(x_i) - f(x_{i-1}) > 0$. Further, since closed
    intervals in $\mathbb{R}$ are compact, by the maximum-minimum
    principle $f([x_{i-1}, x_i])$ is bounded, therefore there must
    exist an $M$ such that
    \[
      f(x_i) < M \implies \sum (f(x_i) - f(x_{i-1})(x_i - x_{i-1}) <
        \sum M(x_i - x_{i-1}) \quad \forall i
      \]
      Which means that for $U(f,P) - L(f, P) < \varepsilon$ to be
      true, $P$ must be selected such that
      \[
        x_i - x_{i-1} < \frac{\varepsilon}{Mn} \quad \forall i
      \]
      So for any $\varepsilon$ we can choose the partition \[
        P = \left\{a + i\frac{b\varepsilon}{M}\right\}_{i=0}^M
      \]
    \end{proof}
    \subsection*{Exercise 3}
    \begin{proof}
      $f$ is continuous, non-negative over the interval $[a,b]$, and
      strictly positive at at least one point $c$. By continuity,
      there exists a $\delta > 0$ such that
      \[
        |x - c| < \delta \implies |f(x) - f(c)| < \varepsilon
      \]
      $\forall \varepsilon > 0$, by setting $\varepsilon :=
      \frac{f(c)}{2}$, we get that there must exist a $\delta > 0$ such that
      \[
        |x - c| < \delta \implies |f(x) - f(c)| < \frac{f(c)}{2}
      \]
      Expanding the absolute value we get that
      \[
        -\frac{f(c)}{2} < f(x) - f(c) < \frac{f(c)}{2} \implies
        \frac{f(c)}{2} < f(x) < \frac{3f(c)}{2}
      \]
      That is, $\forall x \in$ the $\delta$ neighborhood of $c$,
      $f(x) > \frac{f(c)}2$. The integral over that range $[c
      -\delta, c+\delta]$ is bounded above and below:
      \[
        \delta f(c) \le \int_{c-\delta}^{c+\delta} f \le 3\delta f(c)
      \]
      Also, the integral over the ranges to either side of the
      strictly positive range:
      \[
        (c - \delta - a) \inf f(x) \le \int_a^{c-\delta} f \le (c -
        \delta - a)\sup f(x)
      \]
      And since $f(x) \ge 0 \implies \inf f(x) \ge 0$, and $a < c -
      \delta \implies c - \delta - a > 0$. We get that
      \[
        0 \le (c - \delta - a) \inf f(x) \le \int_a^{c-\delta} f
      \]
      Similarily, for the interval $[c+\delta, b]$.
      \[
        (b - c - \delta) \inf f(x) \le \int_{c+\delta}^b f \le (b - c
        - \delta)\sup f(x)
      \]
      And since $f(x) \ge 0 \implies \inf f(x) \ge 0$, and $b > c +
      \delta \implies b - c - \delta > 0$. We get that
      \[
        0 \le (b - c - \delta) \inf f(x) \le \int_{c+\delta}^b f
      \]
      Using property (iv) from theorem 4.6.13:
      \[
        \int_a^b f = \int_a^{c-\delta} f + \int_{c-\delta}^{c+\delta}
        f + \int_{c+\delta}^b f \ge 0 + \delta f(c) + 0 > 0
      \]
      As desired.
    \end{proof}
    \subsection*{Exericse 4}
    \begin{proof}
      Define $G: [a,b] \to \mathbb{R}$ such that $G' = g$. Then by
      the fundamental theorem of Caclulus I,
      \[
        \int_a^b g = G(b) - G(a)
      \]
      Then dividing by $\frac{1}{b-a}$
      \[
        \frac{1}{b-a}\int_a^b g = \frac{G(b) - G(a)}{b-a}
      \]
      By application of the mean value theorem, $\exists c \in [a,b]$ such that
      \begin{align*}
        G'(c) &=  \frac{G(b) - G(a)}{b-a} \\
        g(c) &=  \frac{G(b) - G(a)}{b-a} \\
        g(c) &= \frac1{b-a}\int_a^b g
      \end{align*}
    \end{proof}
    \subsection*{Exercise 5}
    \begin{proof}
      For a fixed $x \in [0, \infty]$, the sequence
      \[
        \frac{nx}{1 + nx^2}
      \]
      is bounded because, $\forall n$,
      \[
        \frac{nx}{1 + nx^2} = \frac{1}{x}\frac{nx^2}{1+nx^2}
      \]
      Since $nx^2 < 1 + nx^2 \implies \frac{nx^2}{1+nx^2} < 1$
      \[
        \frac{nx}{1 + nx^2} < \frac{1}{x}
      \]
      It is also monotone, because, $\forall n$
      \begin{align*}
        \frac{nx}{1 + nx^2} &= \frac{1}{x}\frac{nx^2}{1+nx^2} \\
        &=  \frac{1}{x}\frac{nx^2 + 1 - 1}{1+nx^2} \\
        &= \frac{1}{x}(1 - \frac{1}{1+nx^2})
      \end{align*}
      \begin{align*}
        \frac{(n+1)x}{1+(n+1)x^2} &= \frac{nx + x}{1+ nx^2 + x^2} \\
        &= \frac{1}x\frac{nx^2 + x^2 + 1 - 1}{1 + nx^2 + x^2} \\
        &= \frac{1}x(1 - \frac{1}{1 + nx^2 + x^2})
      \end{align*}
      and since $1 + nx^2 + x^2 \geq 1 + nx^2$
      \[
        \frac{nx}{1+nx^2} \leq \frac{(n+1)x}{1 + (n+1)x^2}
      \]
      So the sequence is bounded and monotone increasing, therefore
      it is convergent $\forall x \in (0, \infty)$ so the family of
      functions are piecewise continuous.
    \end{proof}
    \end{document}
