\documentclass[12pt]{article}

\usepackage[utf8]{inputenc}
\usepackage{latexsym,amsfonts,amssymb,amsthm,amsmath}

\setlength{\parindent}{0in}
\setlength{\oddsidemargin}{0in}
\setlength{\textwidth}{6.5in}
\setlength{\textheight}{8.8in}
\setlength{\topmargin}{0in}
\setlength{\headheight}{18pt}



\title{Homework #2}
\author{N. Ohayon Rozanes}

\begin{document}

\maketitle

\vspace{0.5in}



\subsection*{Exercise 1}
\begin{enumerate}
  \item
\begin{proof}
  Our goal is to find $N$ such that  $ \frac{2s}N + \frac1N^2 < 2  = \frac{2sN + 1}{N^2} < \delta$ 
  \begin{align*}
    \frac{2sN + 1}{N^2} &< \delta \\
                      0 &< -1 - 2sN + \delta N^2  \\ 
  \end{align*}
  Using the quadratic roots of the above inequality:
  \[
    N_0 = \frac{2s + \sqrt{4s^2 + 4\delta}}{2\delta} = \frac{s + \sqrt{s^2 + \delta}}{\delta}
  \]
  The negative root can be discarded since $N$ is strictly positive. So 
  $$\forall N > \frac{s + \sqrt{s^2 + \delta}}{\delta}\implies (s + \frac1N)^2 < 2$$ 
  Therefore $s + \frac1N \in A$. However $s$ is the supremum of $A$, and $s < s + \frac1N$, leading to a contradiction. 
\end{proof}
\item
  \begin{proof}
  Assume $s^2 > 2$. Want to find $N$ such that 
  \[
  (s - \frac1N)^2 > 2
  \]
  This is achieved in a similar manner to part 1. 
  \begin{align*}
    s^2 - \frac{2s}N + \frac1{N^2} &> 2 \\
    s^2 - 2 - \frac{2s}N + \frac1{N^2} &> 0 \\ 
    (s^2 - 2)N^2 - 2sN + 1 &> 0 \\
  \end{align*}
  Which has solution: 
  \[
    N_0 = \frac{2s + \sqrt{4s^2 - 4(s^2 -2)}}{2(s^2 - 2)} = \frac{s + \sqrt{2}}{(s^2 -2)} 
  \]
  So
  \[
    \forall N > \frac{s + \sqrt{2}}{(s^2 -2)} \implies (s - \frac1N)^2  > 2
  \]
  Which would mean that $s - \frac1N$ is an upper bound of $A$, however, since $s > s - \frac1N$ it would be smaller than $s$ which is a contradiction since $s = \sup A$
  \end{proof}
\end{enumerate}

\subsection*{Exercise 2}
\begin{proof}
  Fix first $b \in B$, then $Ab := \{a \cdot b \mid a\in A\}$  then, using the result 
  \[
    cA := \{c \cdot a \mid a \in A\} \quad \implies \quad \sup(cA) = c \cdot \sup A,
  \]
  we get $\sup Ab = b \cdot \sup A \quad \forall b \in B$. Also, for a fixed $a \in A$, $\sup aB = a \sup B \quad \forall a \in A$ \\ 

  Consider the set $C = \{b\sup A \mid b \in B\}$, then 
  \[
    \sup C = \sup \{b \cdot \sup A\} = \sup B \sup A
  \]
  and also: 
  \[
    \sup AB = \sup(Ab) = \sup C = \sup B \sup A
  \]
\end{proof}
\subsection*{Exercise 3}
\begin{proof}
  For set $$A = \bigcap_{n = 1}^{\infty} (0, \frac1n) = \emptyset$$ then $\forall x \in \mathbb{R}, x \notin A$. Also \[
  A_N = \bigcap_{n=1}^N(0, \frac1n) = (0, \frac1N)
  \]
  To find the final open interval of $A$ as $N \rightarrow \infty$, we evaluate the series $\{\frac1n\}_{n=1}^\infty$, since with every increment of $n$, $\frac1n$ halves, we hypothesize that the limit is $0$. To prove this limit, $\forall \varepsilon > 0$ find an $N$ such that $\forall n \geq N, \frac1n < \varepsilon$
  \begin{align*}
    \mid \frac1N - 0\mid &< \varepsilon  \\
    \mid \frac1N \mid &< \varepsilon  \\
    \frac1N &< \varepsilon \\ 
    \frac1\varepsilon &< N
  \end{align*}
  In this case, simply take the next integer larger than $\frac1\varepsilon$ to be $N$. Therefore the final open interval, which is the interval $A$ evaluates to, is $(0,0)$. And sinceopen intervals do not include the endpoints 
  \[
    A = (0,0) = \emptyset
  \]
\end{proof}  
\subsection*{Exercise 4}
\begin{proof}
  For the sequence $r_n$ to converge to real number $a$, it must mean that $\forall \varepsilon > 0 \quad \exists N \in \mathbb{N} : \forall n \geq N \implies \mid r_n - a \mid < \varepsilon$
  \[
    
  \]
\end{proof}
\end{document}

