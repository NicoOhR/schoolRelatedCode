\documentclass[12pt]{article}

\usepackage[utf8]{inputenc}
\usepackage{latexsym,amsfonts,amssymb,amsthm,amsmath}

\setlength{\parindent}{0in}
\setlength{\oddsidemargin}{0in}
\setlength{\textwidth}{6.5in}
\setlength{\textheight}{8.8in}
\setlength{\topmargin}{0in}
\setlength{\headheight}{18pt}
\title{MATH 4301}
\author{N. Ohayon Rozanes}

\begin{document}

\maketitle

\vspace{0.5in}

\subsection*{Exercise 1}
\begin{proof}
  \begin{enumerate}
    \item A geometric series converges as such:
      \[
        \sum r^n = \frac{1}{1-r}
      \]
      For $|r| < 1$, then for
      \[
        \sum r^n = \frac{1}{1-r} = \frac1{1+x^2}
      \]
      It must be the case that $r = -(x^2)$. To show that the series
      \[
        \sum (-x^2)^n
      \]
      Indeed converges, it suffices to show that $|-x^2| = |x^2| <
      1$. Of course,

      \[
        -1 < x < 1 \implies |x| < 1 \implies |x^2| < 1
      \]
      Thus the series of functions desired is
      \[
        \sum f_n(x) = \sum (-x^2)^n = \frac{1}{1+x^2}
      \]
    \item
      recall that $$\arctan(x) = \int_0^x \frac{1}{1+t^2} = \int_0^x f(t)$$
      As was shown in the previous part
      \[
        f(x) = \sum (-x^2)^n \implies \int_0^x f(t) = \int_0^x \sum (-t^2)^n
      \]
      We now show the integrand converges uniformly such that the sum
      and integral can be interchanged. Define $t \in [-x,x]$ for any
      $x \in [-1,1]$
      and $$ g_n(t) = | (-1)^nt^{2n}| = |t^{2n}|$$.
      Define $M_n = |x^{2n}|$ and notice that $\forall n,t, g_n <
      M_n$
      By the $p$-test for convergance, the series $M_n$ converges
      since $|x| < 1$. So by the Weirstress $M$ test, the series
      $g_n$ converges uniformly. So
      \[
        \int_0^x \sum_0^\infty (-t^2)^n = \sum_0^\infty \int_0^x (-t^2)^n
      \]
      By standard integration
      \[
        \sum_0^\infty \int_0^x (-t^2)^n = \sum_0^\infty
        \frac{-x^{2n+1}}{2n+1} = \sum_0^\infty \frac{(-1)^{n}}{2n+1}x^{2n+1}
      \]
      As desired.
  \end{enumerate}
\end{proof}
\subsection*{Exercise 2}
\begin{proof}
  First we calcualte the intergral with respect to $x$
  \begin{align*}
    \int f_n(x) &= \int nx(1-x^2)^n \\
    &= - \frac{n(1-x^2)^{n+1}}{2(n+1)}
  \end{align*}
  When evaluated on the range $0$ to $1$:
  \[
    \int_0^1 f_n = \frac{n}{2(n+1)}
  \]
  Then taking the limit, notice that, $\forall n$
  \begin{align*}
    \frac{n+1}{2(n+1)} \leq \frac{n}{2(n+1)} \leq \frac{n}{2n}\\
    \frac12 \leq \frac{n}{2(n+1)} \leq \frac{1}{2}
  \end{align*}
  so by the squeeze theorem, the sequence converges to $\frac12$. Thus
  \[
    \lim_{n\to\infty}\int_0^1 f_n  = \frac12
  \]
  To check if the sequence conveges uniformly, notice that for any
  fixed $x  \in (0, 1)$
  \[
    0 < (1-x^2)^n < 1
  \]
  and using the standard limit, that for $|a| < 1$,
  $$
  \lim_{n\to\infty} na^n = 0
  $$
  we get that
  \[
    \lim_{n\to\infty}  nx(1 - x^2)^n = 0
  \]
  Thus
  \[
    \int^1_0 \lim_{n\to\infty} f_n = \int^1_0 0 = 0
  \]
  Since
  \[
    \lim_{n\to\infty} \int_0^1 f_n \neq \int^1_0 f_n
  \]
  By theorem 5.4.2, we get that the sequence does not converge uniformly.
\end{proof}
\subsection*{Exercise 3}
\begin{proof}
  \begin{enumerate}
    \item
      $\forall n \in \mathbb{N}$ and $x\in [1,2]$
      \[
        g_n(x) = \frac{x}{(1+x)^{n}}\le
        \frac{2}{(1+1)^{n}} = \frac{1}{2^{n-1}} := M_n
      \]
      Evaluating the convergance of the series of $M_n$ by the ratio test
      \[
        \lim_{n\to\infty}\frac{M_{n+1}}{M_n} =
        \lim_{n\to\infty}\frac{2^{n-1}}{2^{n}}= \frac{1}{2}
      \]
      Hence, since $\sum_0^\infty M_n$ converges in $\mathbb{R}$, by
      the Weirstress $M$ test, $\sum_0^\infty g_n(x)$ converges uniformly.
    \item We have that $\sum g_n$ is uniformly convergant on the
      range $[1,2]$, which implies that the series is pointwise
      convergant. It remains to be proven that $g_n$ is
      differentiable, $g_n'$ is continuous, and that the series
      $g_n'$ converges uniformly. Clearly, both $f(x) = x$ and $h(x)
      = (x+1)^{n}$ are differentiable, thus $g_n = \frac{f}{h}$ is also
      differentiable, and also:
      \begin{align*}
        \left(\frac{f(x)}{h(x)}\right)' &=
        \left(\frac{x}{(1+x)^{n}}\right)' \\
        &= \frac{(1+x)^{n} - n(x+1)^{n-1}}{(1+x)^{2n}} \\
        &= \frac{1}{(1+x)^n} - \frac{n}{(1+x)^{n+1}}
      \end{align*}
      Since $1+x$ is continuous and never $0$ on $[1, 2]$, then
      $(1-x)^{-n}$ is contintuous and so is $(1-x)^{n+1}$, and the
      sum of continuous function is also continuous, so $g_n'$ is
      continuous for all $n$.

      Now we evaluate that the series converges uniformly:
      \begin{align*}
        \sum_{n=0}^\infty \left(\frac{1}{(1+x)^n} -
        \frac{n}{(1+x)^{n+1}}\right) \le
        \sum_{n=0}^\infty \left(\frac{1}{2^n} -
        \frac{n}{2^{n+1}}\right) := \sum_{n=0}^\infty M_n
      \end{align*}
      To show that the above series is convergant:
      \[
        \sum_{n=0}^\infty M_n = \sum_{n=0}^\infty \frac{1}{2^n} -
        \sum_{n=1}^\infty
        \frac{n}{2^n}
      \]
      The first sum converges by the geometric series test. The
      second sum converges by the ratio test
      \begin{align*}
        \lim_{n\to\infty} \frac{(n+1)2^{n}}{2^{n+1}n} &=
        \lim_{n\to\infty} \frac{n+1}{2n} \\
        &= \lim_{n\to\infty} \left(\frac{1}{2} + \frac{1}{2n}\right) \\
        &= \frac12
      \end{align*}
      Thus $\sum M_n$ converges, so $\sum g'_n$
      converges uniformly. All conditions have been verified for
      \[
        \left(\sum g_n\right)' = \sum g_n'
      \]
  \end{enumerate}
\end{proof}
\subsection*{Exericse 4}
\begin{proof}
  \begin{enumerate}
    \item Fix an $x \in \mathbb{R}$, then
      $$
      |g_n| = \left|\frac{x\sin(nx)}{n^3}\right| \le
      \frac{|x|}{n^3}
      $$
      The series
      \[
        \sum_{n=0}^\infty \frac{|x|}{n^3} = |x|\sum_{n=0}^\infty \frac{1}{n^3}
      \]
      Converges by the $p$ series test. Thus for any fixed $x$, the
      series converges pointwise to a function $f$.
    \item From the above, we already have that $\sum g_n$ converges pointwise,
      to show that $f$ is differentiable, we have to show that $g_n$
      is differentiable and that $g'_n$ is continuous, and that
      $\sum g'_n$ converges uniformly. Both $x$ and $sin(nx)$ are
      differentiable $\forall x \in \mathbb{R}$, so their product is
      as well, and is equal to:
      \[
        (x\sin(nx))' = \sin(nx) + xn\cos(nx)
      \]
      Thus $g'_n = \frac1{n^3}(\sin(nx) + nx\cos(nx))$, clearly,
      $\sin(nx), x, \cos(nx)$ are all continuous, thus their
      combinations are continuous as well. Finally, we show that
      \[
        \sum_{n=0}^\infty g'_n
      \]
      converges uniformly. Notice that
      $$
      g'_n = \frac1{n^3} (\sin(nx) + xn\cos(nx)) \le \frac1{n^3}(1 + xn)
      $$
      So for the restricted domain $[-R, R]$, then
      \[
        g'_n \le \frac{1}{n^3}(1+ Rn) := M_n
      \]
      The series $\sum M_n$ converges by the $p$ test:
      \begin{align*}
        \sum_{n=0}^\infty M_n &= \sum_{n=0}^\infty \left(\frac1{n^3}
        + \frac{R}{n^2}\right)\\
        &=  \sum_{n=0}^\infty \frac1{n^3}
        + R\sum_{n=0}^\infty\frac{1}{n^2}\\
      \end{align*}
      So $g'_n$ converges uniformly for $x \in [-R, R]$, since $R$ is
      arbitrary, we can conclude that $\sum g'_n$ converges uniformly
      on $\mathbb{R}$. To find $f'$
      \begin{align*}
        f' &= \left(\sum_{n=0}^\infty g'_n \right) \\
      \end{align*}
  \end{enumerate}
\end{proof}
\subsection*{Exercise 5}
\begin{proof}
  Firstly, computing the $f'_n$ we find that
  \begin{align*}
    f_n'(x) &= \frac{2nx(1+nx^2) - 2nx(nx^2)}{(1+nx^2)^2}\\
    &=  \frac{2nx(1+nx^2 - nx^2)}{(1+nx^2)^2} \\
    &=  \frac{2nx}{(1+nx^2)^2} \\
  \end{align*}
  Fix $x \in (-1, 1)$, then $\forall n \in \mathbb{N}$
  \begin{align*}
    \frac{1}{(1+nx^2)^2} &< \frac{2nx}{(1+nx^2)^2} <
    \frac{2nx}{n^2x^4} = \frac{2}{x^3}\frac{1}n
  \end{align*}
  Since both sequences bounding $f_n'$ converge to $0$, by the squeeze
  theorem, $f_n'$ converges to $0$ as well, proving pointwise convergance.
\end{proof}
\end{document}
