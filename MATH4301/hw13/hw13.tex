\documentclass[12pt]{article}

\usepackage[utf8]{inputenc}
\usepackage{latexsym,amsfonts,amssymb,amsthm,amsmath}

\setlength{\parindent}{0in}
\setlength{\oddsidemargin}{0in}
\setlength{\textwidth}{6.5in}
\setlength{\textheight}{8.8in}
\setlength{\topmargin}{0in}
\setlength{\headheight}{18pt}

\title{MATH 4301}
\author{N. Ohayon Rozanes}

\begin{document}

\maketitle

\vspace{0.5in}

\subsection*{Exercise 1}
\begin{proof}
  \begin{enumerate}
    \item A geometric series converges as such:
      \[
        \sum r^n = \frac{1}{1-r}
      \]
      For $|r| < 1$, then for
      \[
        \sum r^n = \frac{1}{1-r} = \frac1{1+x^2}
      \]
      It must be the case that $r = -(x^2)$. To show that the series
      \[
        \sum (-x^2)^n
      \]
      Indeed converges, it suffices to show that $|-x^2| = |x^2| <
      1$. Of course,

      \[
        -1 < x < 1 \implies |x| < 1 \implies |x^2| < 1
      \]
      Thus the series of functions desired is
      \[
        \sum f_n(x) = \sum (-x^2)^n = \frac{1}{1+x^2}
      \]
    \item
      \begin{align*}
        \left |f(x) -f_k(x) \right | &= \left |\sum_{n=k+1}^\infty
        \frac{(-1)^n}{2n+1}x^{2n+1}\right| \\
        &\le \sum_{n=k+1}^\infty \frac{x^{2n+1}}{2n+1} \\
      \end{align*}
      Using the ratio test.
      \[
        a_{n} := \frac{x^{2n+1}}{2n+1}
      \]
      taking the limit of the ratio
      \[
        \lim_{n\to\infty}\left | \frac{a_{n+1}}{a_n} \right | =
        \lim_{n\to\infty}|x|^2\frac{2n+1}{2n+3} = |x|^2
      \]
      Since $|x| < 1 \implies |x|^2 < 1$, so by the ratio test, the
      series converges, then there exists an $N \in \mathbb{N}$ such that
      \[
        \left|\frac{(-1)^n}{2n+1}x^{2n+1}\right|
        \le \sum_{n=k+1}^\infty \frac{x^{2n+1}}{2n+1} < \varepsilon
      \]
      whenever $k \geq N$, so whenever $k \geq N$
      \[
        |f(x) - f_k(x)| < \varepsilon \quad \forall x \in (-1,1)
      \]
      Which is exactly pointwise convergance.
      % Not sure how to show convergance to arctan though?
  \end{enumerate}
\end{proof}
\subsection*{Exercise 2}
\begin{proof}
  We are given that $f_n \to 0$ pointwise, but for interchange of
  limit and integration, we would like the funtion to be uniformly
  convergent. To demonstrate this: $\exists N \in \mathbb{N}$ such
  that, when $n > N$
  \[
    |f_n(x) - f(x)| < \varepsilon
  \]

  $\forall x \in [0,1]$. Since we know that the function converges
  pointwise to $0$
  \begin{align*}
    |f_n(x) - f(x)| &= |f_n(x) -0|\\
    &= |nx(1-x^2)^n| \\
    &= |nx(1+x)^n(1-x)^n| \\
  \end{align*}
  Since for any limits $a_n \to a$ and $b_n \to b$, $a_nb_n \to ab$,
  it suffices to show that
  \[
    nx(1-x^2)^n \le n(1 - x^2)^n
  \]
  \[
    nx(1-x^2)^n \le \varepsilon
  \]

  By theorem 5.4.2:
  \[
    \lim_{n\to\infty}\int_0^1 nx(1-x^2)^n dx = \int_0^1 f
  \]

\end{proof}
\subsection*{Exercise 3}
\begin{proof}
  \begin{enumerate}
    \item consider the residual of the partial sum and total sum:
      \[
        \left|\sum_{n=1}^\infty \frac{x}{(1+x)^n} - \sum_{n=1}^k
        \frac{x}{(1+x)^n}\right| =
        \left|\sum_{n=k+1}^\infty \frac{x}{(1+x)^n}\right|
      \]
      Which by the ratio test:
  \end{enumerate}
\end{proof}
\subsection*{Exericse 4}
\begin{proof}
\end{proof}
\subsection*{Exercise 5}
\begin{proof}
\end{proof}
\end{document}
