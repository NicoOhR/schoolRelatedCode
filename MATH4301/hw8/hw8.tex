\documentclass[12pt]{article}

\usepackage[utf8]{inputenc}
\usepackage{latexsym,amsfonts,amssymb,amsthm,amsmath}

\setlength{\parindent}{0in}
\setlength{\oddsidemargin}{0in}
\setlength{\textwidth}{6.5in}
\setlength{\textheight}{8.8in}
\setlength{\topmargin}{0in}
\setlength{\headheight}{18pt}



\title{Math 333 Weekly Homework X}
\author{Your Name Here}

\begin{document}

\maketitle

\vspace{0.5in}



\subsection*{Exercise 1}
True
\begin{proof}
  Since $A$ is bounded above, there must exist a $\sup A$. Using the approximation characteritic of the supremum, we can say that $\forall \varepsilon > 0, \exists a \in A$ such that $$\sup - \varepsilon < a$$, so within every $\varepsilon$ ball around $\sup A$, we can find a point in $A$ within the $\varepsilon$ ball of $\sup A$. If this point $a \neq \sup A$ then $\sup A$ is a point in the closure of $A$ not in $A$ and therefore not in the interior of $A$, so $\sup A \in \partial A$. However if $\sup A \in A$, since $\forall a \in A$ and $\forall \varepsilon > 0$  $$\sup A \geq a \implies \sup A + \varepsilon > a$$ then there is always a point in the $\varepsilon$ ball around $\sup A$ is not in $A$. If $\sup A$ is in $A$, then it must be in $\bar A$, but it cannot be in the interior of $A$ since there is a point not in $A$ in an arbitrary $\varepsilon$-ball about it, so it must be in $\partial A$
\end{proof}

\subsection*{Exercise 2}
\begin{proof}
  Let $x \in \bar A$, if $x \in A$, then $x \in A \cup \partial A$ and we're done. Otherwise, $x \notin A$, then $x$ must be an accumulation point of $A$, and by definition 
  $$\mathbb{B}_\varepsilon(x)/\{x\} \cap A \neq \emptyset \implies \mathbb{B}_\varepsilon(x) \cap A \neq \emptyset$$ Further since $x \notin A$, then at least
  \[
    \{x\} \subseteq \mathbb{B}_\varepsilon(x) \cap A^c \implies \mathbb{B}_\varepsilon(x) \cap A^c \neq \emptyset
  \]
  Therefore for a point $x \in \bar A$ not in $A$, $\forall \epsilon > 0, \mathbb{B}_\varepsilon(x)\cap A \neq \emptyset$ and also $\mathbb{B}_\varepsilon \cap A^c \neq \emptyset$ so $x \in \partial A$
\end{proof}
\subsection*{Exercise 3}
\begin{proof}
  $\partial(\partial A) = \overline {\partial A} \setminus \text{Int}(\partial A)$, however since $\partial A$ is closed, and the closure of a closed set is itsefl, $\overline{\partial A} = \partial A$. Therefore $$\partial(\partial A) = \partial A \setminus \text{Int}(\partial A) \implies \partial(\partial A) \subseteq \partial A$$
\end{proof}

\subsection*{Exericse 4}
\begin{enumerate}
  \item 
\begin{proof}
  Using the fact that the square root function is strictly increasing, that is $\forall \delta > 0$, then 
  \[
    \sqrt{x} < \sqrt{x + \delta}
  \]
  and that $|x| = \sqrt{x^2}$, we get that
  \[
    |x^i| = \sqrt{(x^i)^2} \le \sqrt{(x^i)^2 + \delta} \le \sqrt{(x^1)^2 + (x^2)^2... (x^n)^2} = ||x||
  \]
  so $|x^i| \le ||x||$ as desired. For the upper bound, again using the fact that the square root function is strictly increasing, for $x,y,z \in \mathbb{R}, y > z$ 
\[
  \sqrt{x + y} > \sqrt{x + z}
\]
and that $x^m := \max\{x^1, x^2, x^3... x^n\} \ge x^i, \forall i$,
\[
  \sqrt{(x^1)^2 + (x^2)^2 ... + (x^n)^2} \le \sqrt{(x^m)^2 + (x^m)^2 ... (x^m)^2} = \sqrt{n(x^m)^2}  = \sqrt{n}\sqrt{(x^m)^2}
\]
so we get that $||x|| \le \sqrt{n}|x^m|$ as desired.
\end{proof}
\item 
  \begin{proof}
    \begin{enumerate}
      \item $(\implies)$Assume that for all $i$, $x^i_k \to x^i ~ k\to\infty$. Want to show that there exists and $N \in \mathbb{N}$ such that $$|| x_k - x || < \varepsilon \quad \forall k \ge N$$
    expanding the vectors into their individual elements
    \[
      ||(x_k^1, x_k^2...x_k^n) - (x^1, x^2, ... x^n)|| = ||(x_k^1 - x^1), (x_k^2 - x^2),... (x_k^n - x^n)||
    \]
    Which is evaluated as 
    \[
      \sqrt{\sum_{i=1}^n\left( x_k^i - x^i\right)^2} 
    \]
    By assumption $x^i_k \to x^i$, which is to say that $\forall \varepsilon$ there exists $N^i$ such that $|x^i_k - x^i| < \frac{\varepsilon}{\sqrt{n}}$ for all $k > N^i$. Define $N := \max\{ N^1, ,N^2 ... N^n\}$. For all $k > N$ 
    \[
      \sqrt{\sum_{i=1}^n\left( x_k^i - x^i\right)^2}  \le \sqrt{\sum_{i=1}^n (\frac\varepsilon{\sqrt{n}})^2} = \varepsilon
    \] So for any $\varepsilon$, we can choose an $N$ large enough such that $||x_k - x|| < \varepsilon$ as desired.\\ 
    \item 
    $(\Longleftarrow)$ Assume that for arbitrary $\varepsilon > 0 ~ \exists N \in \mathbb{N}$ such that $||x_k - x|| < \varepsilon ~ \forall k > N$. Using the fact that for any vector $x$, the norm of the vector is at least as large as the absolute value of any element of that vector
    $$|x^i_k - x^i| \le ||x_k-x|| < \epsilon$$
    That is to say that any $x^i_k$, the same $N$ for which $||x_k - x|| < \epsilon ~ \forall k > N$ can be chosen such that $|x^i_k - x^i| < \epsilon ~ \forall k > N$
    \end{enumerate}
  \end{proof}
  \subsection*{Exercise 5}
  \begin{enumerate}
    \item Assume that $x_k \to x$ as $k \to \infty$, so $\forall \varepsilon > 0, \exists N \in \mathbb{N}$ such that $||x_k - x|| < \varepsilon$, for $||x_k|| \to ||x||$. Using the reverse triangle inequality, we find that
      \[
        |||x_k|| - ||x||| \le ||x_k - x|| \le \varepsilon
      \]
      So for any $\varepsilon > 0$, choosing the same $N$ for which $||x_k - x|| \le \varepsilon$ will lead to $|||x_k|| - ||x||| \le \varepsilon$, hence $||x_k|| \to ||x||$
    \item A set $F$ is closed if for every convergent sequence $x_k$ defined inside the set, the limit of the sequence $x \in F$. Want to show that $||x|| \le 1$
    Let $\{x_k\} \subset \bar{\mathbb{B}}_1(0)$, and $x_k \to x$. Then by definition $\forall \varepsilon > 0, \exists N \in \mathbb{N}$ such that $$||x_k - x|| < \varepsilon \quad \forall k > N$$
    Since $\forall k, x_k \in \bar{\mathbb{B}}_1(0) \implies ||x_k|| \le 1$, and that $||x|| \to ||x_k||$
    \begin{align*}
      ||x|| &= ||x - x_k + x_k|| \\
            &\le ||x - x_k|| + ||x_k|| \\
            &\le \varepsilon + ||x_k|| \\ 
            &\le \varepsilon + 1
    \end{align*}      
    And since $\varepsilon$ is arbitrary, $||x||\le 1$ so $x$ lies in $\bar{\mathbb{B}}_{1}(0)$
  \end{enumerate}
\end{enumerate}
\end{document}

